\section{Trace Execution}
\label{sec:Execution}

COMPSs Runtime can generate a post-execution trace of the distributed execution of the application. This trace is useful for
performance analysis and diagnosis.

A trace file may contain different events to determine the COMPSs master state, the task execution state or the file-transfers.
Despite the fact that in the current release we do not support file-transfers, we intend to support them in a near future release.

During the execution of the application, an XML file is created at worker nodes to keep track of 
these events. At the end of the execution, all the XML files are merged to get a final trace file.

In the following sections we explain the command used for tracing, how the events are registered, 
in a process called instrumentation, how to visualize the trace file and make a good analysis of 
performance based on the data shown in the trace.

\subsection{Trace Command}
In order to obtain a post-execution trace file the option \textbf{-t}  must be added to the runcompss command. Next we provide an
example of the command execution with the tracing option enabled for the Hmmer java application.
\begin{lstlisting}[language=bash]
compss@bsc:~$ runcompss -t --classpath=/home/compss/workspace_java/hmmerobj/jar/hmmerobj.jar 
                        hmmerobj.HMMPfam 
                        /sharedDisk/Hmmer/smart.HMMs.bin /sharedDisk/Hmmer/256seq 
                        /home/compss/out.txt 2 8 -A 222
\end{lstlisting}
 

\subsection{Application Instrumentation}
The instrumentation is the process that intercepts different events of the application execution 
and keeps log of them. This will cause an overhead in the execution time of the application that 
the user should take into account, but the collected data will be extremely useful for performance 
analysis and diagnosis.

COMPSs Runtime uses the \textit{Extrae} tool to dynamically instrument the application and the \textit{Paraver} tool to visualize
the obtained tracefiles. Both tools are developped at \textit{BSC} and are available in its webpage \url{http://bsc.es} . 

At the worker nodes, in background, \textit{Extrae} keeps track of the events in an intermediate format 
file (with \textit{.mpit} extension). Inside the master node, at the end of the execution, \textit{Extrae} merges the 
intermediate files to get the final trace file, a \textit{Paraver} format file (.prv). See the visualization 
section \ref{sec:Visualization} in this manual for further information about the \textit{Paraver} tool.

When instrumenting the application \textit{Extrae} will output several messages. At the master node, \textit{Extrae} will show up its
initialization at the begining of the execution and the merging process and the paraver generation at the end of the execution. At the
worker nodes \textit{Extrae} will inform about the intermediate files generation every time a task is executed. Next we provide a 
summary of the \textit{stdout} generated by Hmmer java application execution with the trace flag enabled. 
\begin{lstlisting}[language=bash]
----------------- Executing hmmerobj.HMMPfam --------------------------

WARNING: IT Properties file is null. Setting default values
Welcome to Extrae 3.1.1rc (revision 3360 based on extrae/trunk)
Extrae: Warning! EXTRAE_HOME has not been defined!.
Extrae: Generating intermediate files for Paraver traces.
Extrae: Intermediate files will be stored in /home/compss/workspace_java/hmmerobj/jar
Extrae: Tracing buffer can hold 500000 events
Extrae: Tracing mode is set to: Detail.
Extrae: Successfully initiated with 1 tasks

Extrae: Warning! API tries to initialize more than once
Extrae:          Previous initialization was done by API

[   API]  -  Starting COMPSs Runtime v1.3 (build 20150821-1134.rnull)

...
...
...

[   API]  -  No more tasks for app 1
[   API]  -  Getting Result Files 1
[   API]  -  Execution Finished

Extrae: Intermediate raw trace file created : /home/compss/workspace_java/hmmerobj/jar/set-0/TRACE@bsc.0000031637000000000000.mpit
Extrae: Intermediate raw sym file created : /home/compss/workspace_java/hmmerobj/jar/set-0/TRACE@bsc.0000031637000000000000.sym
Extrae: Deallocating memory.
Extrae: Application has ended. Tracing has been terminated.

merger: Output trace format is: Paraver
merger: Extrae 3.1.1rc (revision 3360 based on extrae/trunk)

mpi2prv: Checking for target directory existance... exists, ok!
mpi2prv: Selected output trace format is Paraver
mpi2prv: Stored trace format is Paraver
mpi2prv: Parsing intermediate files
mpi2prv: Removing temporal files... done
mpi2prv: Congratulations! ./trace/hmmerobj.HMMPfam_compss_trace_1440151114.prv has been generated.

------------------------------------------------------------
\end{lstlisting}

For further information about \textit{Extrae} please visit the following site: 
\begin{center}
\url{http://www.bsc.es/computer-science/extrae} 
\end{center}
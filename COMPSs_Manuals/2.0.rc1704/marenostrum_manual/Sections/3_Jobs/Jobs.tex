\section{COMPSs Jobs}
\label{sec:Jobs}

\subsection{Submiting COMPSs jobs}
COMPSs jobs can be easily submited by running the \textbf{enqueue\_compss} command. This command allows to configure any 
\textbf{runcompss} option and some particular queue options such as the queue system, the number of nodes, the wallclock time,
the master working directory, the workers working directory and number of tasks per node.

Next, we provide detailed information about the \textit{enqueue\_compss} command:
\begin{lstlisting}[language=bash]
$ enqueue_compss -h

Usage: enqueue_compss [queue_system_options] [COMPSs_options] 
          application_name application_arguments

* Options:
  General:
    --help, -h                              Print this help message
  
  Queue system configuration:
    --sc_cfg=<name>                         SuperComputer configuration file to use. 
                                            Must exist inside queues/cfgs/
                                            Default: default

  Submission configuration: 
    --exec_time=<minutes>                   Expected execution time of the application (in minutes)
                                            Default: 10
    --num_nodes=<int>                       Number of nodes to use
                                            Default: 2
    --num_switches=<int>                    Maximum number of different switches. Select 0 for no 
                                            restrictions.
                                            Maximum nodes per switch: 18
                                            Only available for at least 4 nodes. 
                                            Default: 0 
    --gpus_per_node=<int>                   Number of desired GPUs per node.
                                            Leave this field empty if your application doesn't use GPUs.
                                            Default: 
    --queue=<name>                          Queue name to submit the job. Depends on the queue system.
                                            For example (MN3): bsc_cs | bsc_debug | debug | interactive
                                            Default: default
    --reservation=<name>                    Reservation to use when submitting the job. 
                                            Default: disabled
    --job_dependency=<jobID>                Postpone job execution until the job dependency has ended.
                                            Default: None
    --storage_home=<string>                 Root installation dir of the storage implementation
                                            Default: null
    --storage_props=<string>                Absolute path of the storage properties file
                                            Mandatory if storage_home is defined

  Launch configuration:
    --tasks_per_node=<int>                  Maximum number of simultaneous tasks running on a node
                                            Default: 16
    --node_memory=<MB>                      Maximum node memory: disabled | <int> (MB)
                                            Default: disabled
    --network=<name>                        Communication network for transfers: default | ethernet 
                                            | infiniband | data.
                                            Default: ethernet
                                              
    --prolog="<string>"                     Task to execute before launching COMPSs (Notice the quotes)
                                            If the task has arguments split them by "," rather than spaces.
                                            This argument can appear multiple times for more than one 
                                            prolog action
                                            Default: Empty
    --epilog="<string>"                     Task to execute after executing the COMPSs application 
                                            (Notice the quotes)
                                            If the task has arguments split them by "," rather than spaces.
                                            This argument can appear multiple times for more than one 
                                            epilog action
                                            Default: Empty

    --master_working_dir=<path>             Working directory of the application
                                            Default: .
    --worker_working_dir=<name | path>      Worker directory. Use: scratch | gpfs | <path>
                                            Default: scratch
                                              
    --worker_in_master_tasks=<int>          Maximum number of tasks that the master node can run as worker.
                                            Cannot exceed tasks_per_node.
                                            Default: 0
    --worker_in_master_memory=<int> MB      Maximum memory in master node assigned to the worker. 
                                            Cannot exceed the node_memory.
                                            Mandatory if worker_in_master_tasks is specified.
                                            Default: disabled
    --jvm_worker_in_master_opts="<string>"  Extra options for the JVM of the COMPSs Worker in the 
                                            Master Node. 
                                            Each option separed by "," and without blank spaces 
                                            (Notice the quotes)
                                            Default: ""

  Runcompss configuration:
    
  Tools enablers:
    --graph=<bool>, --graph, -g             Generation of the complete graph (true/false)
                                            When no value is provided it is set to true
                                            Default: false
    --tracing=<level>, --tracing, -t        Set generation of traces and/or tracing level 
                                            ( [ true | basic ] | advanced | false)
                                            True and basic levels will produce the same traces.
                                            When no value is provided it is set to true
                                            Default: false
    --monitoring=<int>, --monitoring, -m    Period between monitoring samples (milliseconds)
                                            When no value is provided it is set to 2000
                                            Default: 0
    --external_debugger=<int>,
    --external_debugger                     Enables external debugger connection on the specified 
                                            port (or 9999 if empty)
                                            Default: false

  Runtime configuration options:
    --task_execution=<compss|storage>       Task execution under COMPSs or Storage.
                                            Default: compss
    --storage_conf=<path>                   Path to the storage configuration file
                                            Default: None
    --project=<path>                        Path to the project XML file
                                            Default: /opt/COMPSs/Runtime/configuration/xml/
                                            projects/default_project.xml
    --resources=<path>                      Path to the resources XML file
                                            Default: /opt/COMPSs/Runtime/configuration/xml/
                                            resources/default_resources.xml                                                 
    --lang=<name>                           Language of the application (java/c/python)
                                            Default: java
    --summary                               Displays a task execution summary at the end of 
                                            the application execution
                                            Default: false
    --log_level=<level>, --debug, -d        Set the debug level: off | info | debug
                                            Default: off
                                                                                                                                                                                        
  Advanced options:                                                                                                                                                                     
    --extrae_config_file=<path>             Sets a custom extrae config file. Must be in a shared disk
                                            between all COMPSs workers.                                                      
                                            Default: null                                                                                                                               
    --comm=<path>                           Class that implements the adaptor for communications                                                                                        
                                            Supported adaptors: integratedtoolkit.nio.master.NIOAdaptor 
                                                                | integratedtoolkit.gat.master.GATAdaptor                                       
                                            Default: integratedtoolkit.nio.master.NIOAdaptor                                                                                            
    --conn=<path>                           Path of the connector jar/s that should be loaded. 
                                            You can use multiple by splitting with ':'                                               
                                            Supported connectors: compss-{CONN}-connector.jar                                                                                           
                                               (where {CONN} can be: "jclouds", "amazon", "docker", 
                                               "one", "rocci", "vmm", etc.)                                                        
    --scheduler=<path>                      Class that implements the Scheduler for COMPSs                                                                                              
                                            Supported schedulers: 
                                               integratedtoolkit.components.impl.TaskScheduler 
                                               | integratedtoolkit.scheduler.readyscheduler.ReadyScheduler           
                                            Default: 
                                               integratedtoolkit.scheduler.readyscheduler.ReadyScheduler                                                                          
    --library_path=<path>                   Non-standard directories to search for libraries (e.g. Java 
                                            JVM library, Python library, C binding library)                                 
                                            Default: Working Directory                                                                                                                  
    --classpath=<path>                      Path for the application classes / modules                                                                                                  
                                            Default: Working Directory                                                                                                                  
    --appdir=<path>                         Path for the application class folder.                                                                                                      
                                            Default: /home/cramonco
    --base_log_dir=<path>                   Base directory to store COMPSs log files (a .COMPSs/ folder
                                            will be created inside this location)
                                            Default: User home
    --specific_log_dir=<path>               Use a specific directory to store COMPSs log files (the 
                                            folder MUST exist and no sandbox is created)
                                            Warning: Overwrites --base_log_dir option
                                            Default: Disabled
    --uuid=<int>                            Preset an application UUID
                                            Default: Automatic random generation
    --master_name=<string>                  Hostname of the node to run the COMPSs master
                                            Default: 
    --master_port=<int>                     Port to run the COMPSs master communications.
                                            Only for NIO adaptor
                                            Default: [43000,44000]
    --jvm_master_opts="<string>"            Extra options for the COMPSs Master JVM. Each option separed 
                                            by "," and without blank spaces (Notice the quotes)
                                            Default: 
    --jvm_workers_opts="<string>"           Extra options for the COMPSs Workers JVMs. Each option separed 
                                            by "," and without blank spaces (Notice the quotes)
                                            Default: -Xms1024m,-Xmx1024m,-Xmn400m
    --task_count=<int>                      Only for C/Python Bindings. Maximum number of different
                                            functions/methods, invoked from the application, that 
                                            have been selected as tasks
                                            Default: 50
    --pythonpath=<path>                     Additional folders or paths to add to the PYTHONPATH
                                            Default: /home/cramonco
    --PyObject_serialize=<bool>             Only for Python Binding. Enable the object serialization
                                            to string when possible (true/false).
                                            Default: false

* Application name:
    For Java applications:   Fully qualified name of the application
    For C applications:      Path to the master binary
    For Python applications: Path to the .py file containing the main program

* Application arguments:
    Command line arguments to pass to the application. Can be empty.
  
\end{lstlisting}

\subsection{Tracking COMPSs jobs}
When submitting a COMPSs job a temporal file will be created storing the job information. For example:
\begin{lstlisting}[language=bash]
$ enqueue_compss \
  --exec_time=15 \
  --num_nodes=3 \
  --tasks_per_node=16 \
  --master_working_dir=. \
  --worker_working_dir=gpfs \
  --lang=python \
  --log_level=debug \
  <APP> <APP_PARAMETERS>

  
SC Configuration:          default.cfg
Queue:                     default
Reservation:               disabled
Num Nodes:                 3
Num Switches:              0
GPUs per node:             0
Job dependency:            None
Exec-Time:                 00:15
Storage Home:              null
Storage Properties:        null
Other:                     
        --sc_cfg=default.cfg
        --tasks_per_node=16
        --master_working_dir=.
        --worker_working_dir=gpfs
        --lang=python
        --classpath=.
        --library_path=.
        --comm=integratedtoolkit.nio.master.NIOAdaptor
        --tracing=false
        --graph=false
        --pythonpath=.
        <APP> <APP_PARAMETERS>
Temp submit script is: /scratch/tmp/tmp.pBG5yfFxEo

$ cat /scratch/tmp/tmp.pBG5yfFxEo
#!/bin/bash
#
#BSUB -J COMPSs
#BSUB -cwd . 
#BSUB -oo compss-%J.out
#BSUB -eo compss-%J.err
#BSUB -n 3
#BSUB -R "span[ptile=1]"
#BSUB -W 00:15 
...
\end{lstlisting}

In order to trac the jobs state users can run the following command:
\begin{lstlisting}[language=bash]
$ bjobs
JOBID  USER   STAT  QUEUE  FROM_HOST  EXEC_HOST  JOB_NAME  SUBMIT_TIME
XXXX   bscXX  PEND  XX     login1     XX         COMPSs    Month Day Hour
\end{lstlisting}

The specific COMPSs logs are stored under the \textit{~/.COMPSs/} folder; saved as a local \textit{runcompss} execution. For further 
details please check \textit{COMPSs User Manual: Application Execution} available at our webpage \url{http://compss.bsc.es} .

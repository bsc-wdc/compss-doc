\section{Configuration Files}
\label{sec:configuration_files}

The COMPSs runtime has two configuration files: \texttt{resources.xml} and \texttt{project.xml} . 
These files contain information about the execution environment and are completely independent from the application.

For each execution users can load the default configuration files or specify their custom configurations 
by using, respectively, the \texttt{--resources=<absolute\_path\_to\_resources.xml>} and the
\texttt{--project=<absolute\_path\_to\_project.xml>} in the \texttt{runcompss} command. The default files are located 
in the \texttt{/opt/COMPSs/Runtime/configuration/xml/} path. 

Next sections describe in detail the \texttt{resources.xml} and the \texttt{project.xml} files, 
explaining the available options.

\subsection{Resources file}
The \texttt{resources} file provides information about all the available resources that can be used for an execution. 
This file should normally be managed by the system administrators. Its full definition schema can be found at \\
\texttt{/opt/COMPSs/Runtime/configuration/xml/resources/resource\_schema.xsd}. 

For the sake of clarity, users can also check the SVG schema located at \\
\texttt{/opt/COMPSs/Runtime/configuration/xml/resources/resource\_schema.svg}.

This file contains one entry per available resource defining its name and its capabilities. Administrators can define several
resource capabilities (see example in the next listing) but we would like to underline the importance of 
\textbf{ComputingUnits}. This capability represents the number of available cores in the described resource and it is
used to schedule the correct number of tasks. Thus, it becomes essential to define it accordingly to the number of cores 
in the physical resource. 

\begin{lstlisting}[language=xml,moredelim={[is][\textcolor{red}]{@@}{@@}}]
compss@bsc:~$ cat /opt/COMPSs/Runtime/configuration/xml/resources/default_resources.xml
<?xml version="1.0" encoding="UTF-8" standalone="yes"?>
<ResourcesList>
    <ComputeNode Name="localhost">
        <Processor Name="P1">
            @@<ComputingUnits>4</ComputingUnits>@@
            <Architecture>amd64</Architecture>
            <Speed>3.0</Speed>
        </Processor>
        <Processor Name="P2">
            @@<ComputingUnits>2</ComputingUnits>@@
        </Processor>
        <Adaptors>
            <Adaptor Name="integratedtoolkit.nio.master.NIOAdaptor">
                <SubmissionSystem>
                    <Interactive/>
                </SubmissionSystem>
                <Ports>
                    <MinPort>43001</MinPort>
                    <MaxPort>43002</MaxPort>
                </Ports>
            </Adaptor>
        </Adaptors>
        <Memory>
            <Size>16</Size>
        </Memory>
        <Storage>
            <Size>200.0</Size>
        </Storage>
        <OperatingSystem>
            <Type>Linux</Type>
            <Distribution>OpenSUSE</Distribution>
        </OperatingSystem>
        <Software>
            <Application>Java</Application>
            <Application>Python</Application>
        </Software>
    </ComputeNode>
</ResourcesList>
\end{lstlisting}


\subsection{Project file}
The project file provides information about the resources used in a specific execution. Consequently, the resources that 
appear in this file are a subset of the resources described in the \texttt{resources.xml} file. This file, that contains
one entry per worker, is usually edited by the users and changes from execution to execution. Its full definition 
schema can be found at 
\texttt{/opt/COMPSs/Runtime/configuration/xml/projects/project\_schema.xsd}.

For the sake of clarity, users can also check the SVG schema located at \\
\texttt{/opt/COMPSs/Runtime/configuration/xml/projects/project\_schema.xsd}.

We emphasize the importance of correctly defining the following entries:
\begin{description}
 \item [installDir] Indicates the path of the COMPSs installation \textbf{inside the resource} (not necessarily the same 
 than in the local machine).
 \item [User] Indicates the username used to connect via ssh to the resource. This user \textbf{must} have passwordless access to the
 resource (for more information check the \textit{COMPSs Installation Manual} available at our website \url{http://compss.bsc.es}).
 If left empty COMPSs will automatically try to access the resource with the \textbf{same username than the one that lauches 
 the COMPSs main application}.
 \item [LimitOfTasks] The maximum number of tasks that can be simultaneously scheduled to a resource. Considering that a task 
 can use more than one core of a node, this value must be lower or equal to the number of available cores in the resource. 
\end{description}


\begin{lstlisting}[language=xml,moredelim={[is][\textcolor{red}]{@@}{@@}}]
compss@bsc:~$ cat /opt/COMPSs/Runtime/configuration/xml/projects/default_project.xml
<?xml version="1.0" encoding="UTF-8" standalone="yes"?>
<Project>
    <!-- Description for Master Node --> 
    <MasterNode\>

    <!--Description for a physical node-->
    <ComputeNode Name="localhost">
        @@<InstallDir>/opt/COMPSs/</InstallDir>@@
        <WorkingDir>/tmp/Worker/</WorkingDir>
        <Application>
            <AppDir>/home/user/apps/</AppDir>
            <LibraryPath>/usr/lib/</LibraryPath>
            <Classpath>/home/user/apps/jar/example.jar</Classpath>
            <Pythonpath>/home/user/apps/</Pythonpath>
        </Application>
        @@<LimitOfTasks>4</LimitOfTasks>@@
        <Adaptors>
            <Adaptor Name="integratedtoolkit.nio.master.NIOAdaptor">
                <SubmissionSystem>
                    <Interactive/>
                </SubmissionSystem>
                <Ports>
                    <MinPort>43001</MinPort>
                    <MaxPort>43002</MaxPort>
                </Ports>
                @@<User>user</User>@@
            </Adaptor>
        </Adaptors>
    </ComputeNode>
</Project>
\end{lstlisting}
\label{lstlisting:project.xml}


\subsection{Configuration examples}
In the next subsections we provide specific information about the services, shared disks, cluster and cloud configurations and 
several \texttt{project.xml} and \texttt{resources.xml} examples. 


\subsubsection{Services configuration}
To allow COMPSs applications to use WebServices as tasks, the \texttt{resources.xml} can include a special type of resource called 
\textit{Service}. For each WebService it is necessary to specify its wsdl, its name, its namespace and
its port. 
\begin{lstlisting}[language=xml]
<?xml version="1.0" encoding="UTF-8" standalone="yes"?>
<ResourcesList>
    <ComputeNode Name="localhost">
      ...                                                                                                                                                                                                
    </ComputeNode>  
    
    <Service wsdl="http://bscgrid05.bsc.es:20390/hmmerobj/hmmerobj?wsdl">                                                                                                                                          
        <Name>HmmerObjects</Name>                                                                                                                                                                                  
        <Namespace>http://hmmerobj.worker</Namespace>                                                                                                                                                              
        <Port>HmmerObjectsPort</Port>                                                                                                                                                                              
    </Service>                                                                                                                                                                                                     
</ResourcesList>  
\end{lstlisting}

When configuring the \texttt{project.xml} file it is necessary to include the service as a worker by adding an
special entry indicating only the name and the limit of tasks as shown in the following example:
\begin{lstlisting}[language=xml]
<?xml version="1.0" encoding="UTF-8" standalone="yes"?>                                                                                                                                                            
<Project>                                                                                                                                                                                                          
    <MasterNode/>                                                                                                                                                                                                  
    <ComputeNode Name="localhost">                                                                                                                                                                                 
      ...                                                                                                                                                                           
    </ComputeNode>   
    
    <Service wsdl="http://bscgrid05.bsc.es:20390/hmmerobj/hmmerobj?wsdl">                                                                                                                                          
        <LimitOfTasks>2</LimitOfTasks>                                                                                                                                                                             
    </Service>                                                                                                                                                                                                     
</Project> 
\end{lstlisting}

\subsubsection{Cluster and grid configuration (static resources)}
In order to use external resources to execute the applications, the following steps have to be followed:

\begin{enumerate}
 \item Install the \textit{COMPSs Worker} package (or the full \textit{COMPSs Framework} package) on all the new 
 resources following the \textit{Installation manual} available at \url{http://compss.bsc.es} .
 \item Set SSH passwordless access to the rest of the remote resources.
 \item Create the \textit{WorkingDir} directory in the resource (remember this path because it is needed 
 for the \texttt{project.xml} configuration).
 \item Manually deploy the application on each node.
\end{enumerate}

The \texttt{resources.xml} and the \texttt{project.xml} files must be configured accordingly.
Here we provide examples about configuration files for Grid and Cluster environments.

~ \newline

\begin{lstlisting}[language=xml]
<?xml version="1.0" encoding="UTF-8" standalone="yes"?>                                                                                                                                                            
<ResourcesList>                                                                                                                                                                                                    
    <ComputeNode Name="hostname1.domain.es">                                                                                                                                                                       
        <Processor Name="MainProcessor">                                                                                                                                                                           
            <ComputingUnits>4</ComputingUnits>                                                                                                                                                                     
        </Processor>                                                                                                                                                                                               
        <Adaptors>                                                                                                                                                                                                 
            <Adaptor Name="integratedtoolkit.nio.master.NIOAdaptor">                                                                                                                                               
                <SubmissionSystem>
                    <Interactive/>
                </SubmissionSystem>
                <Ports>
                    <MinPort>43001</MinPort>
                    <MaxPort>43002</MaxPort>
                </Ports>
            </Adaptor>
            <Adaptor Name="integratedtoolkit.gat.master.GATAdaptor">
                <SubmissionSystem>
                    <Batch>
                        <Queue>sequential</Queue>
                    </Batch>
                    <Interactive/>
                </SubmissionSystem>
                <BrokerAdaptor>sshtrilead</BrokerAdaptor>
            </Adaptor>
        </Adaptors>
    </ComputeNode>
    
    <ComputeNode Name="hostname2.domain.es">
      ...
    </ComputeNode>
</ResourcesList>
\end{lstlisting}

\newpage

\begin{lstlisting}[language=xml]
<?xml version="1.0" encoding="UTF-8" standalone="yes"?>                                                                                                                                                            
<Project>
    <MasterNode/>
    <ComputeNode Name="hostname1.domain.es">
        <InstallDir>/opt/COMPSs/</InstallDir>
        <WorkingDir>/tmp/COMPSsWorker1/</WorkingDir>
        <User>user</User>
        <LimitOfTasks>2</LimitOfTasks>
    </ComputeNode>
    <ComputeNode Name="hostname2.domain.es">
      ...
    </ComputeNode>
</Project>
\end{lstlisting}


\subsubsection{Shared Disks configuration example}
Configuring shared disks might reduce the amount of data transfers improving the application performance. To configure a 
shared disk the users must:
\begin{enumerate}
 \item Define the shared disk and its capabilities
 \item Add the shared disk and its mountpoint to each worker
 \item Add the shared disk and its mountpoint to the master node
\end{enumerate}

Next example illustrates steps 1 and 2. The \texttt{<SharedDisk>} tag adds a new shared disk named \texttt{sharedDisk0} and the
\texttt{<AttachedDisk>} tag adds the mountpoint of a named shared disk to a specific worker.
\begin{lstlisting}[language=xml]
<?xml version="1.0" encoding="UTF-8" standalone="yes"?>
<ResourcesList>
    <SharedDisk Name="sharedDisk0">
        <Storage>
            <Size>100.0</Size>
            <Type>Persistent</Type>
        </Storage>
    </SharedDisk>
    
    <ComputeNode Name="localhost">
      ...
      <SharedDisks>
        <AttachedDisk Name="sharedDisk0">
          <MountPoint>/tmp/SharedDisk/</MountPoint>
        </AttachedDisk>
      </SharedDisks>
    </ComputeNode>
</ResourcesList>
\end{lstlisting} 

On the other side, to add the shared disk to the \textbf{master node}, the users must edit the \texttt{project.xml} file. Next example
shows how to attach the previous \texttt{sharedDisk0} to the master node:

\newpage

\begin{lstlisting}[language=xml]
<?xml version="1.0" encoding="UTF-8" standalone="yes"?>
<Project>
    <MasterNode>
        <SharedDisks>
            <AttachedDisk Name="sharedDisk0">
                <MountPoint>/home/sharedDisk/</MountPoint>
            </AttachedDisk>
        </SharedDisks>
    </MasterNode>
    
    <ComputeNode Name="localhost">
      ...
    </ComputeNode>
</Project>
\end{lstlisting}

Notice that the \texttt{resources.xml} file can have multiple \texttt{SharedDisk} definitions and that the \texttt{SharedDisks}
tag (either in the \texttt{resources.xml} or in the \texttt{project.xml} files) can have multiple \texttt{AttachedDisk} childrens
to mount several shared disks on the same worker or master. 

~ \newline

\subsubsection{Cloud configuration (dynamic resources)}
In order to use cloud resources to execute the applications, the following steps have to be followed:
\begin{enumerate}
 \item Prepare cloud images with the \textit{COMPSs Worker} package or the full \textit{COMPSs Framework} package installed.
 \item The application will be deployed automatically during execution but the users need to set up the configuration files to
 specify the application files that must be deployed. 
\end{enumerate}

The COMPSs runtime communicates with a cloud manager by means of connectors. Each connector implements 
the interaction of the runtime with a given provider's API, supporting four basic 
operations: ask for the price of a certain VM in the provider, get the time needed to create a VM, 
create a new VM and terminate a VM. This design allows connectors to abstract the runtime from the particular API
of each provider and facilitates the addition of new connectors for other providers.

The \texttt{resources.xml} file must contain one or more \textbf{\texttt{<CloudProvider>}} tags
that include the information about a particular provider, associated to a given connector. The tag \textbf{must} have an
attribute \textbf{Name} to uniquely identify the provider. Next example summarizes the information to be specified by the 
user inside this tag.

\newpage

\begin{lstlisting}[language=xml]
<?xml version="1.0" encoding="UTF-8" standalone="yes"?>
<ResourcesList>
    <CloudProvider Name="PROVIDER_NAME">
        <Endpoint>
            <Server>https://PROVIDER_URL</Server>
            <ConnectorJar>CONNECTOR_JAR</ConnectorJar>
            <ConnectorClass>CONNECTOR_CLASS</ConnectorClass>
        </Endpoint>
        <Images>
            <Image Name="Image1">
                <Adaptors>
                    <Adaptor Name="integratedtoolkit.nio.master.NIOAdaptor">
                        <SubmissionSystem>
                            <Interactive/>
                        </SubmissionSystem>
                        <Ports>
                            <MinPort>43001</MinPort>
                            <MaxPort>43010</MaxPort>
                        </Ports>
                    </Adaptor>
                </Adaptors>
                <OperatingSystem>
                    <Type>Linux</Type>
                </OperatingSystem>
                <Software>
                    <Application>Java</Application>
                </Software>
                <Price>
                    <TimeUnit>100</TimeUnit>
                    <PricePerUnit>36.0</PricePerUnit>
                </Price>
            </Image>
            <Image Name="Image2">
                <Adaptors>
                    <Adaptor Name="integratedtoolkit.nio.master.NIOAdaptor">
                        <SubmissionSystem>
                            <Interactive/>
                        </SubmissionSystem>
                        <Ports>
                            <MinPort>43001</MinPort>
                            <MaxPort>43010</MaxPort>
                        </Ports>
                    </Adaptor>
                </Adaptors>
            </Image>
        </Images>
    
        <InstanceTypes>
            <InstanceType Name="Instance1">
                <Processor Name="P1">
                    <ComputingUnits>4</ComputingUnits>
                    <Architecture>amd64</Architecture>
                    <Speed>3.0</Speed>
                </Processor>
                <Processor Name="P2">
                    <ComputingUnits>4</ComputingUnits>
                </Processor>
                <Memory>
                    <Size>1000.0</Size>
                </Memory>
                <Storage>
                    <Size>2000.0</Size>
                </Storage>
            </InstanceType>
            <InstanceType Name="Instance2">
                <Processor Name="P1">
                    <ComputingUnits>4</ComputingUnits>
                </Processor>
            </InstanceType>
         </InstanceTypes>
  </CloudProvider>
</ResourcesList>
\end{lstlisting}

The \texttt{project.xml} complements the information about a provider listed in the \texttt{resources.xml} file. 
This file can contain a \textbf{\texttt{<Cloud>}} tag where to specify a list of providers, each with a 
\textbf{\texttt{<CloudProvider>}} tag, whose \textbf{name} attribute must match one of the providers in the 
\texttt{resources.xml} file. Thus, the \texttt{project.xml} file \textbf{must} contain a subset of the providers
specified in the \texttt{resources.xml} file. Next example summarizes the information to be specified by the user
inside this \texttt{<Cloud>} tag.

\begin{lstlisting}[language=xml]
<?xml version="1.0" encoding="UTF-8" standalone="yes"?>
<Project>
    <Cloud>
        <InitialVMs>1</InitialVMs>
        <MinimumVMs>1</MinimumVMs>
        <MaximumVMs>4</MaximumVMs>
        <CloudProvider Name="PROVIDER_NAME">
            <LimitOfVMs>4</LimitOfVMs>
            <Properties>
                <Property Context="C1">
                    <Name>P1</Name>
                    <Value>V1</Value>
                </Property>
                <Property>
                    <Name>P2</Name>
                    <Value>V2</Value>
                </Property>
            </Properties>
            
            <Images>
                <Image Name="Image1">
                    <InstallDir>/opt/COMPSs/</InstallDir>
                    <WorkingDir>/tmp/Worker/</WorkingDir>
                    <User>user</User>
                    <Application>
                        <Pythonpath>/home/user/apps/</Pythonpath>
                    </Application>
                    <LimitOfTasks>2</LimitOfTasks>
                    <Package>
                        <Source>/home/user/apps/</Source>
                        <Target>/tmp/Worker/</Target>
                        <IncludedSoftware>
                            <Application>Java</Application>
                            <Application>Python</Application>
                        </IncludedSoftware>
                    </Package>
                    <Package>
                        <Source>/home/user/apps/</Source>
                        <Target>/tmp/Worker/</Target>
                    </Package>
                    <Adaptors>
                        <Adaptor Name="integratedtoolkit.nio.master.NIOAdaptor">
                            <SubmissionSystem>
                                <Interactive/>
                            </SubmissionSystem>
                            <Ports>
                                <MinPort>43001</MinPort>
                                <MaxPort>43010</MaxPort>
                            </Ports>
                        </Adaptor>
                    </Adaptors>
                </Image>
                <Image Name="Image2">
                    <InstallDir>/opt/COMPSs/</InstallDir>
                    <WorkingDir>/tmp/Worker/</WorkingDir>
                </Image>
            </Images>
            <InstanceTypes>
                <InstanceType Name="Instance1"/>
                <InstanceType Name="Instance2"/>
            </InstanceTypes>
        </CloudProvider>
        
        <CloudProvider Name="PROVIDER_NAME2">
            ...
        </CloudProvider>
    </Cloud>
</Project>
\end{lstlisting}

For any connector the Runtime is capable to handle the next list of properties:

\begin{table}[!ht]
\def\arraystretch{1.2}
\centering
\begin{tabularx}{\linewidth}{|l|X|} \hline
	\textbf{Name} &\textbf{Description} \\ \hline
	provider-user & Username to login in the provider\\ \hline
	provider-user-credential & Credential to login in the provider\\ \hline
	time-slot & Time slot\\ \hline
	estimated-creation-time & Estimated VM creation time\\ \hline                    
	max-vm-creation-time & Maximum VM creation time\\ \hline
\end{tabularx}
\caption{Connector supported properties in the \texttt{project.xml} file.}
\label{tab:abstract_connector_properties}
\end{table}

Additionally, for any connector based on SSH, the Runtime automatically handles the next list of properties:

\begin{table}[!ht]
\def\arraystretch{1.2}
\centering
\begin{tabularx}{\linewidth}{|l|X|} \hline
	\textbf{Name} &\textbf{Description} \\ \hline
	vm-user & User to login in the VM\\ \hline
	vm-password & Password to login in the VM\\ \hline
	vm-keypair-name & Name of the Keypair to login in the VM\\ \hline
	vm-keypair-location & Location (in the master) of the Keypair to login in the VM \\ \hline
\end{tabularx}
\caption{Properties supported by any SSH based connector in the \texttt{project.xml} file.}
\label{tab:ssh_connector_properties}
\end{table}

Finally, the next sections provide a more accurate description of each of the currently available connector and its specific properties.

%%%%%%%%%%%%
% ROCCI
%%%%%%%%%%%%
\paragraph{Cloud connectors: rOCCI}
The connector uses the rOCCI binary client\footnote{\url{https://appdb.egi.eu/store/software/rocci.cli}} 
(version newer or equal than 4.2.5) which has to be installed in the node where the COMPSs main 
application is executed.

This connector needs additional files providing details about the resource templates available on 
each provider. This file is located under \\ \texttt{<COMPSs\_INSTALL\_DIR>/configuration/xml/templates} path. Additionally, the user must define the virtual images flavors and instance types offered by each provider; 
thus, when the runtime decides the creation of a VM, the connector selects the appropriate image and 
resource template according to the requirements (in terms of CPU, memory, disk, etc) by invoking the 
rOCCI client through Mixins (heritable classes that override and extend the base templates).

Table \ref{tab:rOCCI_extensions} contains the rOCCI specific properties that must be defined under the \texttt{Provider} tag in
the \texttt{project.xml} file and Table \ref{tab:rOCCI_extensions} contains the specific properties that must be defined
under the \texttt{Instance} tag.

\begin{table}[!ht]
\def\arraystretch{1.2}
\centering
\begin{tabularx}{\linewidth}{|l|X|} \hline
	\textbf{Name} &\textbf{Description} \\ \hline
	auth		& Authentication method, x509 only supported \\ \hline
	user-cred	& Path of the VOMS proxy \\ \hline
	ca-path		& Path to CA certificates directory \\ \hline
	ca-file		& Specific CA filename \\ \hline
	owner		& Optional. Used by the PMES Job-Manager \\ \hline
	jobname		& Optional. Used by the PMES Job-Manager \\ \hline
	timeout		& Maximum command time \\ \hline
	username	& Username to connect to the back-end cloud provider \\ \hline
	password	& Password to connect to the back-end cloud provider \\ \hline
	voms		& Enable VOMS authentication \\ \hline
	media-type	& Media type \\ \hline
	resource	& Resource type \\ \hline
	attributes	& Extra resource attributes for the back-end cloud provider \\ \hline
	context		& Extra context for the back-end cloud provider \\ \hline
	action		& Extra actions for the back-end cloud provider \\ \hline
	mixin		& Mixin definition \\ \hline
	link		& Link \\ \hline
	trigger-action	& Adds a trigger \\ \hline
	log-to		& Redirect command logs \\ \hline
	skip-ca-check	& Skips CA checks \\ \hline
	filter		& Filters command output \\ \hline
	dump-model	& Dumps the internal model \\ \hline
	debug		& Enables the debug mode on the connector commands \\ \hline
	verbose		& Enables the verbose mode on the connector commands \\ \hline
\end{tabularx}
\caption{rOCCI extensions in the \texttt{project.xml} file.}
\label{tab:rOCCI_extensions}
\end{table}

\newpage

\bgroup
  \def\arraystretch{1.5}
  \begin{longtable}{| p{0.25\textwidth} | p{0.7\textwidth} |}
      \hline
      \textbf{Instance} 	& Multiple entries of resource templates. \\ \hline
      Type   			& Name of the resource template. It has to be the same name than in the previous files \\ \hline
      CPU    			& Number of cores \\ \hline
      Memory 			& Size in GB of the available RAM \\ \hline
      Disk   			& Size in GB of the storage \\ \hline
      Price  			& Cost per hour of the instance \\ \hline
      \caption{Configuration of the \texttt{<provider>.xml} templates file.}
      \label{tab:rOCCI_configuration}
  \end{longtable}
\egroup


%%%%%%%%%%%%
% JCLOUDS
%%%%%%%%%%%%
\paragraph{Cloud connectors: JClouds}

The JClouds connector is based on the JClouds API version \textit{1.9.1}. Table \ref{tab:jclouds_extensions} shows the extra available 
options under the \textit{Properties} tag that are used by this connector.

\begin{table}[!ht]
\def\arraystretch{1.2}
\centering
\begin{tabularx}{\linewidth}{|l|X|} \hline
	\textbf{Name} &\textbf{Description} \\ \hline
	provider & Back-end provider to use with JClouds (i.e. aws-ec2) \\ \hline
\end{tabularx}
\caption{JClouds extensions in the \texttt{project.xml} file.}
\label{tab:jclouds_extensions}
\end{table}


%%%%%%%%%%%%
% DOCKER
%%%%%%%%%%%%
\paragraph{Cloud connectors: Docker}

This connector uses a Java API client from \url{https://github.com/docker-java/docker-java}, version \textit{3.0.3}. It has not additional
options. Make sure that the image/s you want to load are pulled before running COMPSs with \texttt{docker pull IMAGE}. Otherwise,
the connectorn will throw an exception.


%%%%%%%%%%%%
% MESOS
%%%%%%%%%%%%
\paragraph{Cloud connectors: Mesos}

The connector uses Mesos version \textit{0.28.2} which has to be installed in the node where the COMPSs main application is executed. 
This connector creates a Mesos framework and it uses Docker images to deploy workers, each one with an own IP address. 

By default it does not use authentication and the timeout timers are set to 3 minutes (180.000 milliseconds). The list of optional
properties available from connector is shown in Table \ref{tab:mesos_extensions}.

\begin{table}[!ht]
\def\arraystretch{1.2}
\centering
\begin{tabularx}{\linewidth}{|l|X|} \hline
	\textbf{Name} &\textbf{Description} \\ \hline
	mesos-checkpoint  & Checkpoint for the framework.\\ \hline
	mesos-authenticate & Uses authentication? (\texttt{true}/\texttt{false}) \\ \hline
	mesos-default-principal & Principal for authentication.\\ \hline
	mesos-default-secret  & Secret for authentication.\\ \hline
	mesos-framework-register-timeout & Timeout to wait for Framework to register.\\ \hline
	mesos-framework-register-timeout-units&  Time units to wait for register.\\ \hline
	mesos-worker-wait-timeout & Timeout to wait for worker to be created.\\ \hline
	mesos-worker-wait-timeout-units  & Time units for waiting creation. \\ \hline
	mesos-worker-kill-timeout  & Number of units to wait for killing a worker. \\ \hline
	mesos-worker-kill-timeout-units  & Time units to wait for killing.\\ \hline
\end{tabularx}
\caption{Mesos extensions in the \texttt{project.xml} file.}
\label{tab:mesos_extensions}
\end{table}

\newpage

\noindent
* TimeUnit avialable values: \texttt{DAYS}, \texttt{HOURS}, \texttt{MICROSECONDS}, \texttt{MILLISECONDS}, \texttt{MINUTES}, 
\texttt{NANOSECONDS}, \texttt{SECONDS}.

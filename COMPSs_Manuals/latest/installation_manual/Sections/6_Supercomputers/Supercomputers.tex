\section{Supercomputers}
\label{sec:Supercomputers}

The COMPSs Framework can be installed in any Supercomputer by installing its packages as in a normal distribution. The packages are
ready to be reallocated so the administrators can choose the right location for the COMPSs installation. \newline

However, if the administrators are not willing to install COMPSs through the packaging system, we also provide a \textbf{COMPSs 
zipped file} containing a pre-build script to easily install COMPSs. Next subsections provide further information about this process.


\subsection{Prerequisites}
\label{subsec:supercomputers_prerequisites}
In order to successfully run the installation script some dependencies must be present on the target machine. Administrators must 
provide the correct installation and environment of the following software:
\begin{itemize}
 \item Autotools
 \item BOOST
 \item Java 8 JRE
\end{itemize}

The following environment variables must be defined:
\begin{itemize}
 \item JAVA\_HOME
 \item BOOST\_CPPFLAGS
\end{itemize}

The tracing system can be enhanced with:
\begin{itemize}
 \item PAPI, which provides support for harware counters
 \item MPI, which speeds up the tracing merge (and enables it for huge traces)
\end{itemize}


\subsection{Installation}
To perform the COMPSs Framework installation please execute the following commands:
\begin{lstlisting}[language=bash]
 # Check out the last COMPSs release
 $ wget http://compss.bsc.es/repo/sc/stable/COMPSs_<version>.tar.gz

 # Unpackage COMPSs
 $ tar -xvzf COMPSs_<version>.tar.gz
 
 # Install COMPSs at your preferred target location
 $ cd COMPSs
 $ ./install <targetDir>
 
 # Clean downloaded files
 $ rm -r COMPSs
 $ rm COMPSs_<version>.tar.gz
\end{lstlisting}

The installation script will create a COMPSs folder inside the given \verb|<targetDir>| so the final COMPSs installation will be placed 
under the \verb|<targetDir>/COMPSs| folder. Please note that if the folder already exists it will be \textbf{automatically erased}.

~ \newline
After completing the previous steps, administrators must ensure that the nodes have passwordless ssh access. If it is not the case,
please contact the COMPSs team at \verb|support-compss@bsc.es|.

~ \newline
The COMPSs package also provides a \textit{compssenv} file that loads the required environment to allow users work more easily
with COMPSs. Thus, after the installation process we recomend to source the \verb|<targetDir>/COMPSs/compssenv| into the 
users \textit{.bashrc}.

~ \newline
Once done, remember to log out and back in again to end the installation process.

\subsection{Configuration}
For queue system executions, COMPSs has a pre-build structure (see Figure \ref{fig:queue_scripts_structure}) to execute 
applications in SuperComputers. For this purpose, users must use the \textit{enqueue\_compss} script provided in the COMPSs installation.
This script has several parameters (see \textit{enqueue\_compss -h}) that allow users to customize their executions for any SuperComputer.

\begin{figure}[h!]
  \centering
    \includegraphics[width=0.6\textwidth]{./Sections/6_Supercomputers/Figures/queue_scripts_structure.png}
    \caption{Structure of COMPSs queue scripts. In Blue user scripts, in Green queue scripts and in Orange system dependant scripts}
    \label{fig:queue_scripts_structure}
\end{figure}

To make this structure works, the administrators must define a configuration file for the queue system 
(under \verb|<targetDir>/COMPSs/scripts/queues/cfgs/QUEUE/QUEUE.cfg|) and a configuration file for the specific SuperComputer parameters
(under \verb|<targetDir>| \verb|/COMPSs/scripts/queues/cfgs/SC_NAME.cfg|). The COMPSs installation already provides queue configurations 
for \textit{LSF} and \textit{SLURM} and several examples for SuperComputer configurations. 

To create a new configuration we recommend to use one of the configurations provided by COMPSs (such as the configuration for the
\textit{MareNostrum III} SuperComputer) or to contact us at \url{support-compss@bsc.es} .

\subsection{Post installation}
To check that COMPSs Framework has been successfully installed you may run:
\begin{lstlisting}[language=bash]
 # Check the COMPSs version
 $ runcompss -v
 COMPSs version <version>
\end{lstlisting}

For queue system executions, COMPSs provides several prebuild queue scripts than can be accessible throgh the \textit{enqueue\_compss}
command. Users can check the available options by running:
\begin{lstlisting}[language=bash]
$ enqueue_compss -h

Usage: enqueue_compss [queue_system_options] [COMPSs_options] 
          application_name [application_arguments]

          
* Options:                                                                                                          
  General:                                                                                                          
    --help, -h                              Print this help message                                                 
           
           
  Queue system configuration:                                                                                       
    --sc_cfg=<name>                         SuperComputer configuration file to use. 
                                            Must exist inside queues/cfgs/ 
                                            Default: default                                                        
             
             
  Submission configuration:                                                                                         
    --exec_time=<minutes>                   Expected execution time of the application (in minutes)                 
                                            Default: 10        
                                            
    --num_nodes=<int>                       Number of nodes to use                                                  
                                            Default: 2    
                                            
    --num_switches=<int>                    Maximum number of different switches. 
                                            Select 0 for no restrictions.     
                                            Maximum nodes per switch: 18                                            
                                            Only available for at least 4 nodes.                                    
                                            Default: 0 
                                            
    --queue=<name>                          Queue name to submit the job. Depends on the queue 
                                            system.
                                            For example (Nord3): bsc_cs | bsc_debug | debug 
                                            | interactive
                                            Default: default
                                            
    --reservation=<name>                    Reservation to use when submitting the job. 
                                            Default: disabled
                                            
    --job_dependency=<jobID>                Postpone job execution until the job dependency 
                                            has ended.
                                            Default: None
                                            
    --storage_home=<string>                 Root installation dir of the storage implementation
                                            Default: null
                                            
    --storage_props=<string>                Absolute path of the storage properties file
                                            Mandatory if storage_home is defined

                                            
                                            
                                            
  Launch configuration:
    --cpus_per_node=<int>                   Available CPU computing units on each node
                                            Default: 16
                                            
    --gpus_per_node=<int>                   Available GPU computing units on each node
                                            Default: 0
                                            
    --max_tasks_per_node=<int>              Maximum number of simultaneous tasks running on a node
                                            Default: -1
                                            
    --node_memory=<MB>                      Maximum node memory: disabled | <int> (MB)
                                            Default: disabled
                                            
    --network=<name>                        Communication network for transfers: 
                                            default | ethernet | infiniband | data.
                                            Default: infiniband
                                              
    --prolog="<string>"                     Task to execute before launching COMPSs (Notice the
                                            quotes). If the task has arguments split them by ","
                                            rather than spaces. 
                                            This argument can appear multiple times for more 
                                            than one prolog action
                                            Default: Empty
                                            
    --epilog="<string>"                     Task to execute after executing the COMPSs application
                                            (Notice the quotes). If the task has arguments split
                                            them by "," rather than spaces.
                                            This argument can appear multiple times for more 
                                            than one epilog action
                                            Default: Empty

    --master_working_dir=<path>             Working directory of the application
                                            Default: .
                                            
    --worker_working_dir=<name | path>      Worker directory. Use: scratch | gpfs | <path>
                                            Default: scratch
                                              
    --worker_in_master_cpus=<int>           Maximum number of CPU computing units that the 
                                            master node can run as worker. 
                                            Cannot exceed cpus_per_node.
                                            Default: 0
                                            
    --worker_in_master_memory=<int> MB      Maximum memory in master node assigned to the worker. 
                                            Cannot exceed the node_memory.
                                            Mandatory if worker_in_master_tasks is specified.
                                            Default: disabled
                                            
    --jvm_worker_in_master_opts="<string>"  Extra options for the JVM of the COMPSs Worker in 
                                            the Master Node. Each option separed by "," and without
                                            blank spaces (Notice the quotes)
                                            Default: Empty
                                            
    --container_image=<path>                Runs the application by means of a singularity 
                                            container image
                                            Default: Empty
                                            
    --container_compss_path=<path>          Path where compss is installed in the Singularity 
                                            container image
                                            Default: /opt/COMPSs

                                            
  Runcompss configuration:

  Tools enablers:
    --graph=<bool>, --graph, -g             Generation of the complete graph (true/false)
                                            When no value is provided it is set to true
                                            Default: false
                                            
    --tracing=<level>, --tracing, -t        Set generation of traces and/or tracing level 
                                            ( [ true | basic ] | advanced | false)
                                            True and basic levels will produce the same traces.
                                            When no value is provided it is set to true
                                            Default: false
                                            
    --monitoring=<int>, --monitoring, -m    Period between monitoring samples (milliseconds)
                                            When no value is provided it is set to 2000
                                            Default: 0
                                            
    --external_debugger=<int>,
    --external_debugger                     Enables external debugger connection on the 
                                            specified port (or 9999 if empty)
                                            Default: false

  Runtime configuration options:
    --task_execution=<compss|storage>       Task execution under COMPSs or Storage.
                                            Default: compss
                                            
    --storage_conf=<path>                   Path to the storage configuration file
                                            Default: None
                                            
    --project=<path>                        Path to the project XML file
                                            Default: default_project.xml
                                            
    --resources=<path>                      Path to the resources XML file
                                            Default: default_resources.xml
                                            
    --lang=<name>                           Language of the application (java/c/python)
                                            Default: Inferred if possible. Otherwise: java
                                            
    --summary                               Displays a task execution summary at the end of 
                                            the application execution
                                            Default: false
                                            
    --log_level=<level>, --debug, -d        Set the debug level: off | info | debug
                                            Default: off

  Advanced options:
    --extrae_config_file=<path>             Sets a custom extrae config file. Must be in a 
                                            shared disk between all COMPSs workers.
                                            Default: null
                                            
    --comm=<ClassName>                      Class that implements the adaptor for communications
                                            Supported adaptors: 
                                            integratedtoolkit.nio.master.NIOAdaptor 
                                            | integratedtoolkit.gat.master.GATAdaptor
                                            Default: integratedtoolkit.nio.master.NIOAdaptor
                                            
    --conn=<className>                      Class that implements the runtime connector for 
                                            the cloud
                                            Supported connectors: 
                                            integratedtoolkit.connectors.DefaultSSHConnector
                                            Default: integratedtoolkit.connectors.DefaultSSHConnector
                                            
    --scheduler=<className>                 Class that implements the Scheduler for COMPSs
                                            Supported schedulers: 
                                            integratedtoolkit.scheduler.fullGraphScheduler
                                            .FullGraphScheduler
                                            | integratedtoolkit.scheduler.fifoScheduler.FIFOScheduler
                                            | integratedtoolkit.scheduler.resourceEmptyScheduler
                                            .ResourceEmptyScheduler
                                            Default: integratedtoolkit.scheduler.loadBalancingScheduler
                                            .LoadBalancingScheduler
                                            
    --library_path=<path>                   Non-standard directories to search for libraries
                                            (e.g. Java JVM library, Python library, C binding
                                            library)
                                            Default: Working Directory
                                            
    --classpath=<path>                      Path for the application classes / modules
                                            Default: Working Directory
                                            
    --appdir=<path>                         Path for the application class folder.
                                            Default: User home
                                            
    --base_log_dir=<path>                   Base directory to store COMPSs log files 
                                            (a .COMPSs/ folder will be created inside this location)
                                            Default: User home
                                            
    --specific_log_dir=<path>               Use a specific directory to store COMPSs log files (the folder
                                            MUST exist and no sandbox is created)
                                            Warning: Overwrites --base_log_dir option
                                            Default: Disabled
                                            
    --uuid=<int>                            Preset an application UUID
                                            Default: Automatic random generation
                                            
    --master_name=<string>                  Hostname of the node to run the COMPSs master
                                            Default: None
                                            
    --master_port=<int>                     Port to run the COMPSs master communications.
                                            Only for NIO adaptor
                                            Default: [43000,44000]
                                            
    --jvm_master_opts="<string>"            Extra options for the COMPSs Master JVM. Each 
                                            option separed by "," and without blank spaces
                                            (Notice the quotes)
                                            Default: Empty
                                            
    --jvm_workers_opts="<string>"           Extra options for the COMPSs Workers JVMs. Each 
                                            option separed by "," and without blank spaces 
                                            (Notice the quotes)
                                            Default: -Xms1024m,-Xmx1024m,-Xmn400m
                                            
    --cpu_affinity="<string>"               Sets the CPU affinity for the workers
                                            Supported options: disabled, automatic, user defined
                                            map of the form "0-8/9,10,11/12-14,15,16"
                                            Default: automatic
                                            
    --gpu_affinity="<string>"               Sets the GPU affinity for the workers
                                            Supported options: disabled, automatic, user defined
                                            map of the form "0-8/9,10,11/12-14,15,16"
                                            Default: automatic
                                            
    --task_count=<int>                      Only for C/Python Bindings. Maximum number of different
                                            functions/methods, invoked from the application, that 
                                            have been selected as tasks
                                            Default: 50
                                            
    --pythonpath=<path>                     Additional folders or paths to add to the PYTHONPATH
                                            Default: User home
                                            
    --PyObject_serialize=<bool>             Only for Python Binding. Enable the object 
                                            serialization to string when possible (true/false).
                                            Default: false


* Application name:
    For Java applications:   Fully qualified name of the application
    For C applications:      Path to the master binary
    For Python applications: Path to the .py file containing the main program

* Application arguments:
    Command line arguments to pass to the application. Can be empty.                                         
\end{lstlisting}

If none of the pre-build queue configurations adapts to your infrastructure (lsf, pbs, slurm, etc.) please contact 
the COMPSs team at \verb|support-compss@bsc.es| to find out a solution.

~ \newline
If you are willing to test the COMPSs Framework installation you can run any of the applications available at our application 
repository \url{https://compss.bsc.es/projects/bar}. We suggest to run the java simple application following the steps listed
inside its \textit{README} file. 

~ \newline
For further information about either the installation or the usage please check the \textit{README} file inside the COMPSs package. 


\section{Pip}
\label{sec:Pip}

\subsection{Prerequisites}
\label{subsec:pip_prerequisites}
In order to be able to install COMPSs and PyCOMPSs with Pip the following requirements must be met:
\begin{enumerate}
 \item Have all the dependencies (excluding the COMPSs packages) mentioned in the section \ref{subsec:packages_dependencies} satisfied. For example, the dependency \verb|graphviz|
 must be satisfied, while the dependency \verb|compss-runtime| can be ignored
 \item Have all the requisites for the supercomputers package satisfied (see section \ref{subsec:supercomputers_prerequisites})
 \item Have PIP installed. PIP can be installed via the \verb|python-pip| package
 \item Have an internet connection
\end{enumerate}
\subsection{Installation}
\label{subsec:pip_installation}
Depending on the machine, the installation command may vary. It must be assured that the Python2.7 pip is being executed. Some of the possible scenarios and their proper installation command are:
\begin{enumerate}
 \item There is more than one Python version installed:
 \begin{lstlisting}[language=bash]
 sudo -E python2.7 -m pip install pycompss -v\end{lstlisting}
 \item There is only Python2.7 installed:
 \begin{lstlisting}[language=bash]
 sudo -E pip install pycompss -v \end{lstlisting}
\end{enumerate}
It is recommended to restart the user session once the installation process has finished. An alternative is to type the following command:
\begin{lstlisting}[language=bash]
source /etc/profile.d/compss.sh\end{lstlisting}
However, this command should be executed in every different terminal during the current user session.
\subsection{Configuration}
\label{subsec:pip_configuration}
The steps mentioned at section \ref{sec:Additional_Configuration} must be done in order to have a functional COMPSs and pyCOMPSs installation.
\subsection{Post installation}
As mentioned in section \ref{subsec:pip_installation}, it is recommended to restart the user session once the installation process has finished.
\section{Special Execution Platforms}
\label{sec:Execution_Platfors}

This section provides information about how to run COMPSs Applications in specific platforms such as \textit{Dockers},
\textit{Chameleon} or \textit{MareNostrum}.


%%%%%%%%%%%%%%%%%%%%%%%%%%%%%%%%%%%%%%%%%
%% DOCKERS
%%%%%%%%%%%%%%%%%%%%%%%%%%%%%%%%%%%%%%%%%
\subsection{Dockers}
BLABLABLA


%%%%%%%%%%%%%%%%%%%%%%%%%%%%%%%%%%%%%%%%%
%% CHAMEMELON
%%%%%%%%%%%%%%%%%%%%%%%%%%%%%%%%%%%%%%%%%
\subsection{Chameleon}

\subsubsection{Introduction}

The Chameleon project is a configurable experimental environment for large-scale cloud research based on a \textit{OpenStack} 
KVM Cloud. With funding from the \textit{National Science Foundation (NSF)}, it provides a large-scale platform to the open research
community allowing them explore transformative concepts in deeply programmable cloud services, design, and core technologies. The 
Chameleon testbed, is deployed at the \textit{University of Chicago} and the \textit{Texas Advanced Computing Center} and consists of
650 multi-core cloud nodes, 5PB of total disk space, and leverage 100 Gbps connection between the sites. 

The project is led by the \textit{Computation Institute} at the \textit{University of Chicago} and partners from the \textit{Texas 
Advanced Computing Center} at the \textit{University of Texas} at Austin, the \textit{International Center for Advanced Internet 
Research} at \textit{Northwestern University}, the \textit{Ohio State University}, and \textit{University of Texas} at \textit{San
Antoni}, comprising a highly qualified and experienced team. The team includes members from the \textit{NSF} supported 
\textit{FutureGrid} project and from the \textit{GENI} community, both forerunners of the \textit{NSFCloud} solicitation under 
which this project is funded. Chameleon will also sets of partnerships with commercial and academic clouds, such as \textit{Rackspace},
\textit{CERN} and \textit{Open Science Data Cloud (OSDC)}.

For more information please check \url{https://www.chameleoncloud.org/} .

\subsubsection{Execution}
Currently, COMPSs can only handle the Chameleon infrastructure as a cluster (deployed inside a lease). Next, we provide the steps
needed to execute COMPSs applications at Chameleon:

\begin{itemize}
 \item Make a lease reservation with 1 minimum node (for the COMPSs master instance) and a maximum number of nodes equal to the
 number of COMPSs workers needed plus one
 \item Instantiate the master image (based on the published image \textit{COMPSs\_1.3\_CC-CentOS7})
 \item Attach a public IP and login to the master instance (the instance is correctly contextualized for COMPSs executions if you
 see a COMPSs login banner)
 \item Run the \textit{chameleon\_cluster\_setup} script and fill the information when prompted (you will be asked for the name of the
 master instance, the reservation id and number of workers). This scripts may take several minutes since it sets up the all cluster.
 \item Execute your COMPSs applications normally using the \textit{runcompss} script
\end{itemize}

As an example you can check this video \url{https://www.youtube.com/watch?v=BrQ6anPHjAU} performing a full setup and 
execution of a COMPSs application at Chameleon.


%%%%%%%%%%%%%%%%%%%%%%%%%%%%%%%%%%%%%%%%%
%% SuperComputers
%%%%%%%%%%%%%%%%%%%%%%%%%%%%%%%%%%%%%%%%%
\subsection{SuperComputers}

To maintain the portability between different environments, COMPSs has a pre-build structure (see Figure 
\ref{fig:queue_scripts_structure}) to execute applications in SuperComputers. For this purpose, users must use 
the \textit{enqueue\_compss} script provided in the COMPSs installation. This script has several parameters (see 
\textit{enqueue\_compss -h}) that allow users to customize their executions for any SuperComputer.

\begin{figure}[h!]
  \centering
    \includegraphics[width=0.2\textwidth]{./Sections/5_Execution_Platforms/Figures/queue_scripts_structure.jpeg}
    \caption{Structure of COMPSs queue scripts. In Blue general scripts, in Green system dependant scripts}
    \label{fig:queue_scripts_structure}
\end{figure}

To make this structure work, the administrators must define a submit and a launch script that are system dependant. To develop this
scripts the current COMPSs installation provides scripts for the \textit{MareNostrum III} SuperComputer that can be used as an 
example (based on LSF) or you can contact us at \url{support-compss@bsc.es} .

\subsubsection{MareNostrum III}

For information about how to submit COMPSs applications at MareNostrum III (BSC) please refer to the \textit{COMPSs at BSC} manual 
available at \url{http://compss.bsc.es/releases/compss/latest/docs/COMPSs_MareNostrum_Manual.pdf} .

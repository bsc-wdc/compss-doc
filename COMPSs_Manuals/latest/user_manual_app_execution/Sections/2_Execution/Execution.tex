\section{Executing COMPSs applications}
\label{sec:Executing}


\subsection{Prerequisites}
\label{subsec:prerequisites}
Prerequisites vary depending on the application's code language: for Java applications the users need to have
a \textbf{jar archive} containing all the application classes, for Python applications there are no requirements and for
C/C\texttt{++} applications the code must have been previously compiled by using the \textit{buildapp} command.

For further information about how to develop COMPSs applications please refer to the \textit{COMPSs User 
Manual: Application development guide} available at the \url{http://compss.bsc.es/} webpage.


\subsection{Runcompss command}
\label{subsec:runcompss}
COMPSs applications are executed using the \textbf{runcompss} command:
\begin{lstlisting}[language=bash]
compss@bsc:~$ runcompss [options] application_name [application_arguments]
\end{lstlisting}

The application name must be the fully qualified name of the application in Java, the path to the \textit{.py} file
containing the main program in Python and the path to the master binary in C/C\texttt{++}. 

The application arguments are the ones passed as command line to main application. This parameter can be empty. 

The \texttt{runcompss} command allows the users to customize a COMPSs execution by specifying different options.
For clarity purposes, parameters are grouped in \textit{Runtime configuration}, \textit{Tools enablers}
and \textit{Advanced options}.

\begin{lstlisting}[language=bash]
compss@bsc:~$ runcompss -h

Usage: runcompss [options] application_name application_arguments

* Options:
  General:
    --help, -h                              Print this help message

    --opts                                  Show available options

    --version, -v                           Print COMPSs version
    
  Tools enablers:
    --graph=<bool>, --graph, -g             Generation of the complete graph (true/false)
                                            When no value is provided it is set to true
                                            Default: false
    --tracing=<level>, --tracing, -t        Set generation of traces and/or tracing level 
                                            ( [ true | basic ] | advanced | false)
                                            True and basic levels will produce the same traces.
                                            When no value is provided it is set to true
                                            Default: false
    --monitoring=<int>, --monitoring, -m    Period between monitoring samples (milliseconds)
                                            When no value is provided it is set to 2000
                                            Default: 0
    --external_debugger=<int>,
    --external_debugger                     Enables external debugger connection on the specified 
                                            port (or 9999 if empty)
                                            Default: false

  Runtime configuration options:
    --task_execution=<compss|storage>       Task execution under COMPSs or Storage.
                                            Default: compss
    --storage_conf=<path>                   Path to the storage configuration file
                                            Default: None
    --project=<path>                        Path to the project XML file
                                            Default: /opt/COMPSs/Runtime/configuration/xml/
                                            projects/default_project.xml
    --resources=<path>                      Path to the resources XML file
                                            Default: /opt/COMPSs/Runtime/configuration/xml/
                                            resources/default_resources.xml                                                 
    --lang=<name>                           Language of the application (java/c/python)
                                            Default: java
    --summary                               Displays a task execution summary at the end of 
                                            the application execution
                                            Default: false
    --log_level=<level>, --debug, -d        Set the debug level: off | info | debug
                                            Default: off
                                                                                                                                                                                        
  Advanced options:                                                                                                                                                                     
    --extrae_config_file=<path>             Sets a custom extrae config file. Must be in a shared disk
                                            between all COMPSs workers.                                                      
                                            Default: null                                                                                                                               
    --comm=<path>                           Class that implements the adaptor for communications                                                                                        
                                            Supported adaptors: integratedtoolkit.nio.master.NIOAdaptor 
                                                                | integratedtoolkit.gat.master.GATAdaptor                                       
                                            Default: integratedtoolkit.nio.master.NIOAdaptor                                                                                            
    --conn=<path>                           Path of the connector jar/s that should be loaded. 
                                            You can use multiple by splitting with ':'                                               
                                            Supported connectors: compss-{CONN}-connector.jar                                                                                           
                                               (where {CONN} can be: "jclouds", "amazon", "docker", 
                                               "one", "rocci", "vmm", etc.)                                                        
    --scheduler=<path>                      Class that implements the Scheduler for COMPSs                                                                                              
                                            Supported schedulers: 
                                               integratedtoolkit.components.impl.TaskScheduler 
                                               | integratedtoolkit.scheduler.readyscheduler.ReadyScheduler           
                                            Default: 
                                               integratedtoolkit.scheduler.readyscheduler.ReadyScheduler                                                                          
    --library_path=<path>                   Non-standard directories to search for libraries (e.g. Java 
                                            JVM library, Python library, C binding library)                                 
                                            Default: Working Directory                                                                                                                  
    --classpath=<path>                      Path for the application classes / modules                                                                                                  
                                            Default: Working Directory                                                                                                                  
    --appdir=<path>                         Path for the application class folder.                                                                                                      
                                            Default: /home/cramonco
    --base_log_dir=<path>                   Base directory to store COMPSs log files (a .COMPSs/ folder
                                            will be created inside this location)
                                            Default: User home
    --specific_log_dir=<path>               Use a specific directory to store COMPSs log files (the 
                                            folder MUST exist and no sandbox is created)
                                            Warning: Overwrites --base_log_dir option
                                            Default: Disabled
    --uuid=<int>                            Preset an application UUID
                                            Default: Automatic random generation
    --master_name=<string>                  Hostname of the node to run the COMPSs master
                                            Default: 
    --master_port=<int>                     Port to run the COMPSs master communications.
                                            Only for NIO adaptor
                                            Default: [43000,44000]
    --jvm_master_opts="<string>"            Extra options for the COMPSs Master JVM. Each option separed 
                                            by "," and without blank spaces (Notice the quotes)
                                            Default: 
    --jvm_workers_opts="<string>"           Extra options for the COMPSs Workers JVMs. Each option separed 
                                            by "," and without blank spaces (Notice the quotes)
                                            Default: -Xms1024m,-Xmx1024m,-Xmn400m
    --task_count=<int>                      Only for C/Python Bindings. Maximum number of different
                                            functions/methods, invoked from the application, that 
                                            have been selected as tasks
                                            Default: 50
    --pythonpath=<path>                     Additional folders or paths to add to the PYTHONPATH
                                            Default: /home/cramonco
    --PyObject_serialize=<bool>             Only for Python Binding. Enable the object serialization
                                            to string when possible (true/false).
                                            Default: false

* Application name:
    For Java applications:   Fully qualified name of the application
    For C applications:      Path to the master binary
    For Python applications: Path to the .py file containing the main program

* Application arguments:
    Command line arguments to pass to the application. Can be empty.

\end{lstlisting}

\subsection{Running a COMPSs application}
\label{subsec:running_compss}
Before running COMPSs applications the application files \textbf{must} be in the \textbf{CLASSPATH}.
Thus, when launching a COMPSs application, users can manually pre-set the \textbf{CLASSPATH} environment variable
or can add the \texttt{--classpath} option to the \texttt{runcompss} command.

The next three sections provide specific information for launching COMPSs applications developed in different code languages (Java, Python and 
C/C\texttt{++}). For clarity purposes, we will use the \textit{Simple} application (developed in Java, Python and C\texttt{++}) available in the COMPSs
Virtual Machine or at \url{https://compss.bsc.es/projects/bar} webpage. This application takes an integer as input
parameter and increases it by one unit using a task. For further details about the codes please refer to the \textit{Sample 
Applications} document available at \url{http://compss.bsc.es} .

\subsubsection{Running Java applications}
A Java COMPSs application can be launched through the following command:
\begin{lstlisting}[language=bash]
compss@bsc:~$ cd tutorial_apps/java/simple/jar/
compss@bsc:~/tutorial_apps/java/simple/jar$ runcompss simple.Simple <initial_number>
\end{lstlisting}

\begin{lstlisting}[language=bash]
compss@bsc:~/tutorial_apps/java/simple/jar$ runcompss simple.Simple 1
[  INFO] Using default execution type: compss
[  INFO] Using default location for project file: /opt/COMPSs/Runtime/configuration/xml/projects/default_project.xml
[  INFO] Using default location for resources file: /opt/COMPSs/Runtime/configuration/xml/resources/default_resources.xml
[  INFO] Using default language: java

----------------- Executing simple.Simple --------------------------

WARNING: IT Properties file is null. Setting default values
[(1066)    API]  -  Starting COMPSs Runtime v<version>
Initial counter value is 1
Final counter value is 2
[(4740)    API]  -  Execution Finished

------------------------------------------------------------
\end{lstlisting}

In this first execution we use the default value of the \texttt{--classpath} option to automatically add the jar
file to the classpath (by executing runcompss in the directory which contains the jar file). However,
we can explicitly do this by exporting the \textbf{CLASSPATH} variable or by providing the 
\texttt{--classpath} value. Next, we provide two more ways to perform the same execution:

\begin{lstlisting}[language=bash]
compss@bsc:~$ export CLASSPATH=$CLASSPATH:/home/compss/tutorial_apps/java/simple/jar/simple.jar
compss@bsc:~$ runcompss simple.Simple <initial_number>
\end{lstlisting}

\begin{lstlisting}[language=bash]
compss@bsc:~$ runcompss --classpath=/home/compss/tutorial_apps/java/simple/jar/simple.jar 
                        simple.Simple <initial_number>
\end{lstlisting}


\subsubsection{Running Python applications}
To launch a COMPSs Python application users have to provide the \texttt{--lang=python} option to the runcompss command. If the extension
of the main file is a regular Python extension (\verb|.py| or \verb|.pyc|) the \textit{runcompss} command can also infer the 
application language without specifying the \textit{lang} flag. 

\begin{lstlisting}[language=bash]
compss@bsc:~$ cd tutorial_apps/python/simple/
compss@bsc:~/tutorial_apps/python/simple$ runcompss --lang=python ./simple.py <initial_number>
\end{lstlisting}

\begin{lstlisting}[language=bash]
compss@bsc:~/tutorial_apps/python/simple$ runcompss simple.py 1
[  INFO] Using default execution type: compss
[  INFO] Using default location for project file: /opt/COMPSs/Runtime/configuration/xml/projects/default_project.xml
[  INFO] Using default location for resources file: /opt/COMPSs/Runtime/configuration/xml/resources/default_resources.xml
[  INFO] Inferred PYTHON language

----------------- Executing simple.py --------------------------

WARNING: IT Properties file is null. Setting default values
[(616)    API]  -  Starting COMPSs Runtime v<version>
Initial counter value is 1
Final counter value is 2
[(4297)    API]  -  Execution Finished

------------------------------------------------------------
\end{lstlisting}


\subsubsection{Running C/C\texttt{++} applications}
To launch a COMPSs C/C\texttt{++} application users have to compile the C/C\texttt{++} application by means of the \texttt{buildapp} command.
For further information please refer to the \textit{COMPSs User Manual: Application development guide} document available at
\url{http://compss.bsc.es} . Once complied, the \texttt{--lang=c} option must be provided to the runcompss command. If the main file is 
a C / C++ binary the \textit{runcompss} command can also infer the application language without specifying the \textit{lang} flag. 

\begin{lstlisting}[language=bash]
compss@bsc:~$ cd tutorial_apps/c/simple/
compss@bsc:~/tutorial_apps/c/simple$ runcompss --lang=c simple <initial_number>
\end{lstlisting}

\begin{lstlisting}[language=bash]
compss@bsc:~/tutorial_apps/c/simple$ runcompss ~/tutorial_apps/c/simple/master/simple 1
[  INFO] Using default execution type: compss
[  INFO] Using default location for project file: /opt/COMPSs/Runtime/configuration/xml/projects/default_project.xml
[  INFO] Using default location for resources file: /opt/COMPSs/Runtime/configuration/xml/resources/default_resources.xml
[  INFO] Inferred C/C++ language

----------------- Executing simple --------------------------

JVM_OPTIONS_FILE: /tmp/tmp.ItT1tQfKgP
IT_HOME: /opt/COMPSs
Args: 1
 
WARNING: IT Properties file is null. Setting default values
[(650)    API]  -  Starting COMPSs Runtime v<version>
Initial counter value is 1
[   BINDING]  -  @compss_wait_on  -  Entry.filename: counter
[   BINDING]  -  @compss_wait_on  -  Runtime filename: d1v2_1497432831496.IT
Final counter value is 2
[(4222)    API]  -  Execution Finished

------------------------------------------------------------
\end{lstlisting}


\subsection{Additional configurations}

The COMPSs runtime has two configuration files: \texttt{resources.xml} and \texttt{project.xml} . 
These files contain information about the execution environment and are completely independent from the application.

For each execution users can load the default configuration files or specify their custom configurations 
by using, respectively, the \texttt{--resources=<absolute\_path\_to\_resources.xml>} and the
\texttt{--project=<absolute\_path\_to\_project.xml>} in the \texttt{runcompss} command. The default files are located 
in the \texttt{/opt/COMPSs/Runtime/configuration/xml/} path. 
Users can manually edit these files or can use the \textit{Eclipse IDE} tool developed for COMPSs. For further 
information about the \textit{Eclipse IDE} please refer to Section \ref{subsec:IDE}. 

~\newline

For further details please check the \textit{Configuration Files} Section inside the \textit{COMPSs Installation and Administration Manual}
available at \url{http://compss.bsc.es/releases/compss/latest/docs/COMPSs_Installation_Manual.pdf} .


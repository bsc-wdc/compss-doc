\section{Java}
\label{sec:Java}

This section illustrates the steps to develop a Java COMPSs application, to compile and to execute it. The \textit{Simple} application will be used as reference code. 
The user is required to select a set of methods, invoked in the sequential application, that will be run as remote tasks on the available resources.

\subsection{Programming Model}
A COMPSs application is composed of three parts:
\begin{itemize}
 \item \textbf{Main application code:} the code that is executed sequentially and contains the calls to the user-selected methods 
 that will be executed by the COMPSs runtime as asynchronous parallel tasks.
 \item \textbf{Remote methods code:} the implementation of the tasks.
 \item \textbf{Java annotated interface:} It declares the methods to be run as remote tasks along with metadata information needed by the runtime to properly schedule the tasks.
\end{itemize}

The main application file name has to be the same of the main class and starts with capital
letter, in this case it is \textbf{Simple.java}. The Java annotated interface filename is \textit{application name+Itf.java}, in this case it is \textbf{SimpleItf.java}. And the code that implements the remote tasks is defined in the \textit{application name + Impl.java} file, in this case it is \textbf{SimpleImpl.java}.

All code examples are in the $/home/compss/workspace\_java/$ folder of the development environment.


\subsubsection{Main application code}

In COMPSs the user's application code is kept unchanged, no API calls need to be included in the main
application code in order to run the selected tasks on the nodes.

The COMPSs runtime is in charge of replacing the invocations to the user-selected methods with the
creation of remote tasks also taking care of the access to files where required. 
Let's consider the Simple application example that takes an integer as input parameter and increases it by one unit.

The main application code of Simple app ({\bf Simple.java}) is executed sequentially until the call to the
{\bf increment()} method. COMPSs, as mentioned above, replaces the call to this method with the generation of a remote task that will be executed on an available node.

\begin{lstlisting}[language=java]
package simple;

import java.io.FileInputStream;
import java.io.FileOutputStream;
import java.io.IOException;
import simple.SimpleImpl;

public class Simple {

  public static void main(String[] args) {
    String counterName = "counter";
    int initialValue = args[0];

//------------------------------------------------------------//
//Creation of the file which will contain the counter variable//
//------------------------------------------------------------//
    try {
       FileOutputStream fos = new FileOutputStream(counterName);
       fos.write(initialValue);
       System.out.println("Initial counter value is "
                           +initialValue);
       fos.close();
    }catch(IOException ioe) {
       ioe.printStackTrace();
    }
    
    //----------------------------------------------//
    //           Execution of the program           //
    //----------------------------------------------//
    
    %*{\bf SimpleImpl.increment(counterName);}*)
    
    //----------------------------------------------//
    //    Reading from an object stored in a File   //
    //----------------------------------------------//
    try {
       FileInputStream fis = new FileInputStream(counterName);
       System.out.println("Final counter value is "+fis.read());
       fis.close();
    }catch(IOException ioe) {
       ioe.printStackTrace();
    }
  }
}
\end{lstlisting}


\subsubsection{Remote methods code}

The following code contains the implementation of the remote method of the {\it Simple} application ({\bf SimpleImpl.java})
that will be executed remotely by COMPSs.

\begin{lstlisting}[language=java]
package simple;

import  java.io.FileInputStream;
import  java.io.FileOutputStream;
import  java.io.IOException;
import  java.io.FileNotFoundException;

public class SimpleImpl {
  public static void increment(String counterFile) {
    try{
      FileInputStream fis = new FileInputStream(counterFile);
      int count = fis.read();
      fis.close();
      
      FileOutputStream fos = new FileOutputStream(counterFile);
      fos.write(++count);
      fos.close();
    }catch(FileNotFoundException fnfe){
      fnfe.printStackTrace();
    }catch(IOException ioe){
      ioe.printStackTrace();
    }
  }
}
\end{lstlisting}


\subsubsection{Java annotated interface}
The Java interface is used to declare the methods to be executed remotely along with Java annotations that
specify the necessary metadata about the tasks. The metadata can be of three different types:

\begin{enumerate}
 \item For each parameter of a method, the data type (currently {\it File} type, primitive types and the {\it String} type are supported) and its directions (IN, OUT or INOUT).
 \item The Java class that contains the code of the method.
 \item The constraints that a given resource must fulfill to execute the method, such as the number of processors or main memory size.
\end{enumerate}

A complete and detailed explanation of the usage of the metadata includes:

\begin{itemize}
 \item {\bf Method-level Metadata:} for each selected method, the following metadata has to be defined:
       \begin{itemize}
         \item {\bf @Method:} Mandatory. It specifies the class that implements the method.
		\begin{itemize}
			\item \textbf{isModifier} True if the method modifies the implicit object, false otherwise.
			\item \textbf{priority} True if the task takes priority and false otherwise. This parameter is used by the COMPSs scheduler. 
		\end{itemize}
         \item {\bf @Constraints:} The user can specify the capabilities that a resource must have in order
               to run a method. For example, in a cloud execution the COMPSs runtime creates a VM that fulfils the
               specified requirements in order to perform the execution. A full description of the supported constraints
               can be found in Table \ref{tab:constraints} in Section \ref{sec:Constraints}.
       \end{itemize}

 \item {\bf Parameter-level Metadata (@Parameter):} for each parameter and method, the user must define:
       \begin{itemize}
        \item {\bf Direction:} {\it Direction.IN, Direction.INOUT or Direction.OUT}
        \item {\bf Type:} COMPSs supports the following types for task parameters:
              \begin{itemize}
               \item {\bf Basic types:} {\it Type.BOOLEAN, Type.CHAR, Type.BYTE, Type.SHORT, Type.INT, Type.LONG,
                     Type.FLOAT, Type.DOUBLE}. They can only have {\bf IN} direction, since primitive types in Java are
                     always passed by value.
               \item {\bf String:} {\it Type.STRING}. It can only have {\bf IN} direction, since Java Strings are immutable.
               \item {\bf File:} {\it Type.FILE}. It can have any direction (IN, OUT or INOUT). The real Java type associated
                     with a FILE parameter is a String that contains the path to the file. However, if the user specifies
                     a parameter as a FILE, COMPSs will treat it as such.
               \item {\bf Object:} {\it Type.Object}. It can have any direction (IN, OUT or INOUT).
              \end{itemize}
        \item {\bf Return type:} Any object or a generic class object. In this case the direction is always OUT.
\colorComment{Basic types are also supported as return types. However, we do not recommend to use them because they cause an implicit synchronization}
        \item {\bf Method modifiers:} the method has to be {\bf STATIC}.
       \end{itemize}

 \item \textbf{Service-level Metadata:} for each selected service, the following metadata has to be defined:
       \begin{itemize}
         \item \textbf{@Service:} Mandatory. It specifies the service properties. 
		\begin{itemize}
			\item \textbf{namespace} Mandatory. Service namespace
			\item \textbf{name} Mandatory. Service name.
			\item \textbf{port} Mandatory. Service port.
			\item \textbf{operation} Operation type. 
			\item \textbf{priority} True if the service takes priority, false otherwise. This parameter is used by the COMPSs scheduler. 
		\end{itemize}
	\end{itemize}
\end{itemize}

    
The Java annotated interface of the Simple app example (SimpleItf.java) includes the description of the {\it Increment()} method metadata.
The method interface contains a single input parameter, a string containing a path to the file counterFile. In this example
there are constraints on the minimum number of processors and minimum memory size needed to run the
method.

\begin{lstlisting}[language=java]
package simple;

import  integratedtoolkit.types.annotations.Constraints;
import  integratedtoolkit.types.annotations.Method;
import  integratedtoolkit.types.annotations.Parameter;
import  integratedtoolkit.types.annotations.Parameter.Direction;
import  integratedtoolkit.types.annotations.Parameter.Type;

public interface SimpleItf {

  %*{\bf @Constraints(processorCPUCount = 1, memoryPhysicalSize = 0.3f) }*)
  %*{\bf @Method(declaringClass = "simple.SimpleImpl") }*)
  void increment(
      %*{\bf @Parameter(type = Type.FILE, direction = Direction.INOUT) }*)
      String file
  );

}
\end{lstlisting}


\subsubsection{Alternative method implementations}
Since version 1.2, the COMPSs programming model allows developers to define sets of alternative implementations of the same method in the Java
annotated interface. 
The following code depicts an example where the developer sorts an integer array using two different
methods: merge sort and quick sort that are respectively hosted in the {\it packagepath.Mergesort} and
{\it packagepath.Quicksort} classes.

\begin{lstlisting}[language=java]
%*{\bf @Method(declaringClass = \{ "packagepath.Mergesort", }*)
                         %*{\bf "packagepath.Quicksort" \}) }*)
                           
void sort(
    %*{\bf @Parameter(type = Type.OBJECT, direction = Direction.INOUT) }*)
    int[] array
);
\end{lstlisting}

As depicted in the example, the name and parameters of all the implementations must coincide; the only
difference is the class where the method is implemented. This is reflected in the attribute declaringClass of
the @Method annotation. Instead of stating that the method is implemented in a single class, the
programmer can define an array of declaring classes for the method.

As independent remote methods, the sets of equivalent methods might have common restrictions to be
fulfilled by the resource hosting the execution. Or even, each implementation can have specific constraints.
Through the @Constraints annotation, developers can specify the common constraints for a whole set of
methods. In the following example only one core is required to run the method of both sorting algorithms.

\begin{lstlisting}[language=java]
%*{\bf @Constraints(processorCoreCount = 1) }*)
%*{\bf @Method(declaringClass = \{ "packagepath.Mergesort", }*) 
                         %*{\bf "packagepath.Quicksort"\}) }*)

void sort(
    %*{\bf @Parameter(type = Type.OBJECT, direction = Direction.INOUT) }*)
    int[] array
);
\end{lstlisting}

However, these sorting algorithms have different memory consumption, thus each algorithm might require a
specific amount of memory and that should be stated in the implementation constraints. For this purpose, the
developer can add a @Multiconstraints annotation containing the specific constraints for each
implementation. Since the Mergesort has a higher memory consumption than the quicksort, the following
example sets a requirement of 1 core and 2GB of memory for the mergesort implementation and 1 core and
500MB of memory for the quicksort.

\begin{lstlisting}[language=java]
%*{\bf @Constraints(processorCoreCount = 1) }*)
%*{\bf @MultiConstraints(\{ }*)
        %*{\bf @Constraints(memoryPhysicalSize= (float)2.0), }*)
        %*{\bf @Constraints(memoryPhysicalSize= (float)0.5)\}) }*)
%*{\bf @Method(declaringClass = \{ "packagepath.Mergesort",  }*)
                         %*{\bf "packagepath.Quicksort"\}) }*)
void sort(
    %*{\bf @Parameter(type = Type.OBJECT, direction = Direction.INOUT) }*)
    int[] array
);
\end{lstlisting}

\subsection{Application Compilation}
A COMPSs Java application needs to be packaged in a \textit{jar} file containing the class files of the main code, of the methods implementations and of the \textit{Itf} annotation.
Next we provide a set of commands to compile the Java Simple application detailed at the \textit{COMPSs Sample Applications} 
available at our website \url{http://compss.bsc.es} .
\begin{lstlisting}[language=bash]
compss@bsc:~$ cd workspace_java/simple/src/main/java/simple/
compss@bsc:~/workspace_java/simple/src/main/java/simple$ javac *.java
compss@bsc:~/workspace_java/simple/src/main/java/simple$ cd ..
compss@bsc:~/workspace_java/simple/src/main/java$ jar cf simple.jar simple/
compss@bsc:~/workspace_java/simple/src/main/java$ mv ./simple.jar ../../../jar/
\end{lstlisting}

In order to properly compile the code, the CLASSPATH variable has to contain the path of the \textit{compss-engine.jar} package. The default COMPSs
installation automatically add this package to the CLASSPATH; please check that your environment variable CLASSPATH contains the \textit{compss-engine.jar} location by running the following command:
\begin{lstlisting}[language=bash]
$ echo $CLASSPATH | grep compss-engine
\end{lstlisting}
If the result of the previous command is empty it means that you are missing the \textit{compss-engine.jar} package in your classpath. 
We recommend to automatically load the variable by editing the \textit{.bashrc} file:
\begin{lstlisting}[language=bash]
$ echo "# COMPSs variables for Java compilation" >> ~/.bashrc
$ echo "export CLASSPATH=$CLASSPATH:/opt/COMPSs/Runtime/compss-engine.jar" >> ~/.bashrc
\end{lstlisting}

If you are using an IDE (such as Eclipse or NetBeans) we recommend you to add the \textit{compss-engine.jar} file as an external file 
to the project. The \textit{compss-engine.jar} file is available at your current COMPSs installation under the path \textit{/opt/COMPSs/Runtime/compss-engine.jar}.

\colorComment{Please notice that if you have performed a custom installation, the location of the package can be different.}

An Integrated Development Environment for Eclipse is also available to simplify the development, compilation,
deployment and execution COMPSs applications. For further information about the \textit{COMPSs IDE} please refer to the \textit{COMPSs IDE
User Guide} available at \url{http://compss.bsc.es} .


\subsection{Application Execution}
A Java COMPSs application is executed through the \textit{runcompss} script. An example of an invocation of the script is:

\begin{lstlisting}[language=bash]
compss@bsc:~$ runcompss --classpath=/home/compss/workspace_java/simple/jar/simple.jar 
                        simple.Simple 1
\end{lstlisting}

A comprehensive description of the \textit{runcompss} command is available in the \textit{COMPSs User Manual: Application
Execution} document available at \url{http://compss.bsc.es} . 
\newline
~ \newline

In addition to Java, COMPSs supports the execution of applications written in other languages by 
means of bindings. A binding manages the interaction of the no-Java application with the COMPSs 
Java runtime, providing the necessary language translation.

The next sections describe the Python and C/C++ language bindings offered by COMPSs.

\section{PAPI: Hardware Counters}
\label{sec:papi}

The applications instrumentation supports hardware counters through the performance API (PAPI). In order to use it, PAPI needs to be present on the machine before installing
COMPSs. 

During COMPSs installation it is possible to check if PAPI has been detected in the Extrae config report:

\begin{lstlisting}[language=bash]
Package configuration for Extrae 3.3.0 based on extrae/trunk rev. 3966:
-----------------------
Installation prefix: /opt/COMPSs/Dependencies/extrae
Cross compilation: no
...
...
...

Performance counters: yes
  Performance API: PAPI
  PAPI home: /usr
  Sampling support: yes
\end{lstlisting}


PAPI installation and requirements depend on the OS. On Ubuntu 14.04 it is available under textit{papi-tools} package; on OpenSuse textit{papi} and textit{papi-dev}.
For more information check \url{https://icl.cs.utk.edu/projects/papi/wiki/Installing_PAPI}.

Extrae only supports 8 active hardware counters at the same time. Both basic and advanced mode have the same default counters list:

\begin{description}
 \item [PAPI\_TOT\_INS] Instructions completed
 \item [PAPI\_TOT\_CYC] Total cycles
 \item [PAPI\_LD\_INS] Load instructions
 \item [PAPI\_SR\_INS] Store instructions
 \item [PAPI\_BR\_UCN] Unconditional branch instructions
 \item [PAPI\_BR\_CN] Conditional branch instructions
 \item [PAPI\_VEC\_SP] Single precision vector/SIMD instructions
 \item [RESOURCE\_STALLS] Cycles Allocation is stalled due to Resource Related reason
\end{description}

The XML config file contains a secondary set of counters. In order to activate it just change the \textit{starting-set-distribution} from 2 to 1 under the \textit{cpu} tag. The second set provides the following information:

\begin{description}
 \item [PAPI\_TOT\_INS] Instructions completed
 \item [PAPI\_TOT\_CYC] Total cycles
 \item [PAPI\_L1\_DCM] Level 1 data cache misses
 \item [PAPI\_L2\_DCM] Level 2 data cache misses
 \item [PAPI\_L3\_TCM] Level 3 cache misses
 \item [PAPI\_FP\_INS] Floating point instructions
\end{description}


To further customize the tracked counters, modify the XML to suit your needs. To find the available PAPI counters on a given computer issue the command \textit{papi\_avail -a}. For more information about Extrae's XML configuration refer to \url{https://www.bsc.es/computer-sciences/performance-tools/trace-generation/extrae/extrae-user-guide}.

\section{Paraver: configurations}
\label{sec:configs}

Table \ref{tab:paraver_configs} provides information about the different pre-build configurations that we distribute
with COMPSs and that can be found under the \textit{/opt/COMPSs/Dependencies/paraver/cfgs/} folder.


\bgroup
  \def\arraystretch{1.5}
  \begin{table}[h]
    \begin{center}
      \begin{tabular}{| p{0.45\textwidth} | p{0.45\textwidth} |}
	\hline
	$2dp\_runtime\_state.cfg$		& 2D plot of runtime state \\ \hline
	$2dp\_tasks.cfg$			& 2D plot of tasks duration \\ \hline
	$3dh\_duration\_runtime.cfg$		& 3D Histogram of runtime execution \\ \hline
	$3dh\_duration\_tasks.cfg$		& 3D Histogram of tasks duration \\ \hline
	$compss\_runtime.cfg$ 			& Shows COMPSs Runtime events (master and workers) \\ \hline
	$compss\_tasks\_and\_runtime.cfg$ 	& Shows COMPSs Runtime events (master and workers) and tasks execution \\ \hline
	$compss\_tasks.cfg$ 			& Shows tasks execution \\ \hline
	$compss\_tasks\_numbers.cfg$ 		& Shows tasks execution by task id \\ \hline
	$compss\_tasks\_transfers.cfg$ 		& Shows transfer requests and tasks by id \\ \hline
	$compss\_data\_transfers.cfg$ 		& Shows data transfers for each task's parameter \\ \hline
	$Interval\_between\_runtime.cfg$ 	& Interval between runtime events \\ \hline
	$thread\_cpu.cfg$			& Shows the initial executing CPU. \\ \hline
      \end{tabular}
      \caption{Available paraver configurations for COMPSs Applications}
      \label{tab:paraver_configs}
    \end{center}
  \end{table}
\egroup

\section{Custom Events in Python}

Users can emit custom events inside their python \textbf{tasks}. Thanks to the fact that python isn't a compiled language, users can emit events inside their own tasks using the available extrae instrumentation object because it is already imported.
\\
\\
To emit an event just use the call \textit{pyextrae.event(type, id)} or \textit{pyextrae.eventandcounters(type, id)} if you also want to emit PAPI hardware counters. It is recommended to use a type number higher than 8000050 in order to avoid type's conflicts. This events will appear automatically on the generated trace. In order to visualize them, take, for example, \textit{compss\_runtime.cfg} and go to Window Properties -> Filter -> Events -> Event Type and change the value labeled \textit{Types} for your custom events type.
\\
\\
More information and examples of common python usage can be found under the default directory \textit{/opt/COMPSs/Dependencies/extrae/share/examples/PYTHON}.

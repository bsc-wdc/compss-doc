\documentclass[a4paper,12pt]{article}

\usepackage[utf8]{inputenc}
\usepackage[english]{babel}
\usepackage{hyperref}
\usepackage{fontenc}
\usepackage{graphicx}
\usepackage{makeidx}
\usepackage{color}
\usepackage{multirow}
\usepackage{tabularx}
\usepackage{longtable}
\usepackage{url}
\usepackage{titlesec}
\usepackage{listings}
\usepackage{xcolor}
\usepackage{colortbl}
\usepackage{geometry}
\usepackage{pdflscape}
\usepackage[toc,page,title]{appendix}

\geometry{
    a4paper,
    left=25mm,
    right=25mm,
    top=30mm,
    bottom=30mm,
 }

%%%%%%%%%%%%%%%%%%%%%%%%%%%%%
%%%%%%% CONFIGURATION %%%%%%%
%%%%%%%%%%%%%%%%%%%%%%%%%%%%%
\input{configuration.tex}
\title{COMP Superscalar}
\author{Tracing Manual}
\def \compssversion {2.4.rc1811}

\makeindex

%%%%%%%%%%%%%%%%%%%%%%%%%%%%%
%%%%%%%% DOCUMENT %%%%%%%%%%%
%%%%%%%%%%%%%%%%%%%%%%%%%%%%%
\begin{document}

  %%%%%%%%%%%% TITLE PAGE %%%%%%%%%%%%%
  \hypersetup{pageanchor=false}
  \begin{titlepage} 
    \begin{center} 
      \includegraphics[width=0.3\textwidth]{./Figures/Logos/degradado-naranja-compss.jpg}~\\[1cm] 
      \textsc{\LARGE COMP Superscalar}\\[1.5cm] 
      
      \HRule \\[0.4cm] 
      { \huge \bfseries COMPSs Tracing Manual \\[0.4cm] }
      \HRule \\[1.5cm] 

      { \large \textsc{Version: \compssversion}} \\[0.3cm]
      { \large \today } 
      
      \vfill 
      % Bottom of the page
      \includegraphics[width=0.5\textwidth]{./Figures/bsc_280.jpg}~\\[1cm]
    \end{center} 
  \end{titlepage}
  \hypersetup{pageanchor=true}
  
  %%%%%%%% REFERENCE NOTES %%%%%%%%%%
  {
  
    This manual only provides information about the COMPSs tracing system. Specifically, it illustrates how to run COMPSs applications
    with tracing (using Extrae tool \footnote{For more information: \url{https://www.bsc.es/computer-sciences/extrae}}) and how to visualize, 
    interpret and analyze the obtained traces (using Paraver tool\footnote{For more information: \url{https://www.bsc.es/computer-sciences/performance-tools/paraver}]}).
    \newline

    For further information about the application execution, please refer to the \textit{COMPSs User Manual: Application execution
    guide} available at \url{http://compss.bsc.es/releases/compss/latest/docs/COMPSs_User_Manual_App_Exec.pdf}.
    \newline
    
    For further information about the application development, please refer to the \textit{COMPSs User Manual: Application development
    guide} available at \url{http://compss.bsc.es/releases/compss/latest/docs/COMPSs_User_Manual_App_Development.pdf}.

  }
  
  %%%%%%%% TABLE OF CONTENTS %%%%%%%%%%
  \pagenumbering{roman}
  \setcounter{tocdepth}{2}
  \tableofcontents
  \listoffigures
  \listoftables
    
  \newpage

  %%%%%%%%%%%%% CONTENTS %%%%%%%%%%%%%%
  \pagenumbering{arabic}
    
  \section{COMP Superscalar (COMPSs)}
\label{sec:Introduction}

COMP Superscalar (COMPSs) is a programming model which aims to ease the 
development of applications for distributed infrastructures, such as Clusters, 
Grids and Clouds. COMP Superscalar also features a runtime system that exploits 
the inherent parallelism of applications at execution time.

For the sake of programming productivity, the COMPSs model has four key 
characteristics:

\begin{itemize}
 
 \item  {\bf Sequential programming:} COMPSs programmers do not need to deal 
 with the typical duties of parallelization and distribution, such as thread 
 creation and synchronization, data distribution, messaging or fault tolerance. 
 thus eliminating most of the difficulties of concurrent/distributed programming.
 A task is a method or a service called from the application code that is intended to be spawned asynchronously and possibly run in parallel with other tasks on a set of resources, instead of locally and sequentially.
 
 \item  {\bf Infrastructure unaware:} COMPSs offers a model that abstracts the 
 application from the underlying infrastructure. Hence, COMPSs 
 programs do not include any detail that could tie them to a particular 
 platform, like deployment or resource management. This makes applications 
 portable between infrastructures with diverse characteristics.
 
 \item  {\bf Standard programming languages:} COMPSs natively supports Java applications, but also offers language bindings for Python and 
 C/C++ applications. 
 
 \item  {\bf No APIs:} In the case of COMPSs applications in Java, the model 
 does not require to use any special API call, pragma or construct in the 
 application; everything is standard Java syntax and libraries. As 
 regards the Python and C/C++ bindings, a small set of API calls should be used 
 on the COMPSs applications.

\end{itemize}


  
  \section{Tracing}
\label{sec:tracing}

COMPSs Runtime has a built-in instrumentation system to generate post-execution tracefiles of the applications' execution. The tracefiles contain different events representing the COMPSs master state, the tasks' execution state, and the data transfers (transfers' information is only available when using NIO adaptor), and are useful for both visual and numerical performance analysis and diagnosis. The instrumentation process essentially intercepts and logs different events, so it adds overhead to the execution time of the application.


The tracing system uses Extrae to generate tracefiles of the execution that, in turn, can be visualized with Paraver. Both tools are developed and maintained by the Performance Tools team of the BSC and are available on its web page 
\url{http://www.bsc.es/computer-sciences/performance-tools}. 


For each worker node and the master, Extrae keeps track of the events in an intermediate format file (with \textit{.mpit} extension). At the end of the execution, all intermediate files are gathered and merged with Extrae's \textit{mpi2prv} command in order to create the final tracefile, a Paraver format file (.prv). See the visualization section \ref{sec:Visualization} of this manual for further information about the Paraver tool.


When instrumentation is activated, Extrae outputs several messages corresponding to the tracing initialization, intermediate files' creation, and
the merging process. 


At present time, COMPSs tracing features two execution modes:


\begin{description}
\item [Basic,] aimed at COMPSs applications developers
\item [Advanced,] for COMPSs developers and users with access to its source code or custom installations
\end{description}


Next sections describe the information provided by each mode and how to use them.


\subsection{Basic Mode}

This mode is aimed at COMPSs' apps users and developers. It instruments computing threads and some management resources providing information about tasks' executions, 
data transfers, and hardware counters if PAPI is available (see PAPI counters appendix \ref{sec:papi} for more info). 

\subsubsection{Usage}

In order to activate basic tracing one needs to provide one of the following arguments to the execution command:

\begin{itemize}
 \item -t
 \item --tracing
 \item --tracing=basic
 \item --tracing=true
\end{itemize}


\noindent Examples given:

\begin{lstlisting}[language=bash]
runcompss --tracing application_name application_args
\end{lstlisting}

\noindent Figure \ref{fig:basic_trace} was generated as follows:


\begin{lstlisting}[language=bash]
runcompss \
     --lang=java \
     --tracing \
     --classpath=/path/to/jar/kmeans.jar \
     kmeans.KMeans
\end{lstlisting}

When tracing is activated, Extrae generates additional output to help the user ensure that instrumentation is turned on and working without issues. On basic mode this is the output users should see when tracing is working correctly:

\begin{lstlisting}[language=bash]
*** RUNNING JAVA APPLICATION KMEANS
Resolved: /path/to/jar/kmeans.jar:

----------------- Executing kmeans.Kmeans --------------------------

Extrae: WARNING!
Extrae: WARNING! XML parser version and property 'xml-parser-id' do not match. Check the XML file. Trying to proceed...
Extrae: WARNING!
Extrae: xml-parser-id found 'Id: xml-parse.c 3682 2015-11-26 14:32:27Z harald $' when expecting 'Id: xml-parse.c 3918 2016-03-11 14:59:01Z harald $'.
Welcome to Extrae 3.3.0 (revision 3966 based on extrae/trunk)
Extrae: Parsing the configuration file (/opt/COMPSs/Runtime/scripts/user/../../configuration/xml/tracing/extrae_basic.xml) begins
Extrae: Tracing package is located on /opt/COMPSs/Dependencies/extrae/
Extrae: Generating intermediate files for Paraver traces.
Extrae: Warning! change-at-time time units not specified. Using seconds
Extrae: PAPI domain set to ALL for HWC set 1
Extrae: HWC set 1 contains following counters < PAPI_TOT_INS (0x80000032) PAPI_TOT_CYC (0x8000003b) PAPI_L2_DCM (0x80000002) PAPI_L3_TCM (0x80000008) > - never changes
Extrae: Warning! change-at-time time units not specified. Using seconds
WARNING: IT Properties file is null. Setting default values
[   API]  -  Deploying COMPSs Runtime v1.4 (build 20160412-1147.r2040)
[   API]  -  Tracing is activated
[   API]  -  Starting COMPSs Runtime v1.4 (build 20160412-1147.r2040)
...
...
...
merger: Output trace format is: Paraver
merger: Extrae 3.3.0 (revision 3966 based on extrae/trunk)
mpi2prv: Assigned nodes < Marginis >
mpi2prv: Assigned size per processor < <1 Mbyte >
mpi2prv: File set-0/TRACE@Marginis.0000001904000000000000.mpit is object 1.1.1 on node Marginis assigned to processor 0
mpi2prv: File set-0/TRACE@Marginis.0000001904000000000001.mpit is object 1.1.2 on node Marginis assigned to processor 0
mpi2prv: File set-0/TRACE@Marginis.0000001904000000000002.mpit is object 1.1.3 on node Marginis assigned to processor 0
mpi2prv: File set-0/TRACE@Marginis.0000001980000001000000.mpit is object 1.2.1 on node Marginis assigned to processor 0
mpi2prv: File set-0/TRACE@Marginis.0000001980000001000001.mpit is object 1.2.2 on node Marginis assigned to processor 0
mpi2prv: File set-0/TRACE@Marginis.0000001980000001000002.mpit is object 1.2.3 on node Marginis assigned to processor 0
mpi2prv: File set-0/TRACE@Marginis.0000001980000001000003.mpit is object 1.2.4 on node Marginis assigned to processor 0
mpi2prv: File set-0/TRACE@Marginis.0000001980000001000004.mpit is object 1.2.5 on node Marginis assigned to processor 0
mpi2prv: Time synchronization has been turned off
mpi2prv: A total of 9 symbols were imported from TRACE.sym file
mpi2prv: 0 function symbols imported
mpi2prv: 9 HWC counter descriptions imported
mpi2prv: Checking for target directory existance... exists, ok!
mpi2prv: Selected output trace format is Paraver
mpi2prv: Stored trace format is Paraver
mpi2prv: Searching synchronization points... done
mpi2prv: Time Synchronization disabled.
mpi2prv: Circular buffer enabled at tracing time? NO
mpi2prv: Parsing intermediate files
mpi2prv: Progress 1 of 2 ... 5% 10% 15% 20% 25% 30% 35% 40% 45% 50% 55% 60% 65% 70% 75% 80% 85% 90% 95% done
mpi2prv: Processor 0 succeeded to translate its assigned files
mpi2prv: Elapsed time translating files: 0 hours 0 minutes 0 seconds
mpi2prv: Elapsed time sorting addresses: 0 hours 0 minutes 0 seconds
mpi2prv: Generating tracefile (intermediate buffers of 838848 events)
         This process can take a while. Please, be patient.
mpi2prv: Progress 2 of 2 ... 5% 10% 15% 20% 25% 30% 35% 40% 45% 50% 55% 60% 65% 70% 75% 80% 85% 90% 95% done
mpi2prv: Warning! Clock accuracy seems to be in microseconds instead of nanoseconds.
mpi2prv: Elapsed time merge step: 0 hours 0 minutes 0 seconds
mpi2prv: Resulting tracefile occupies 991743 bytes
mpi2prv: Removing temporal files... done
mpi2prv: Elapsed time removing temporal files: 0 hours 0 minutes 0 seconds
mpi2prv: Congratulations! ./trace/kmeans.Kmeans_compss_trace_1460456106.prv has been generated.
[   API]  -  Execution Finished
Extrae: Tracing buffer can hold 100000 events
Extrae: Circular buffer disabled.
Extrae: Warning! <dynamic-memory> tag will be ignored. This library does support instrumenting dynamic memory calls.
Extrae: Warning! <input-output> tag will be ignored. This library does support instrumenting I/O calls.
Extrae: Dynamic memory instrumentation is disabled.
Extrae: Basic I/O memory instrumentation is disabled.
Extrae: Parsing the configuration file (/opt/COMPSs/Runtime/scripts/user/../../configuration/xml/tracing/extrae_basic.xml) has ended
Extrae: Intermediate traces will be stored in /home/kurtz/compss/tests_local/app10
Extrae: Tracing mode is set to: Detail.
Extrae: Successfully initiated with 1 tasks and 1 threads

\end{lstlisting}

It contains diverse information about the tracing, for example, Extrae version used (3.3.0), the XML configuration file used (extrae\_basic.xml), the amount of 
threads instrumented (objects through 1.1.1 to 1.2.5), available hardware counters ( PAPI\_TOT\_INS (0x80000032) ... PAPI\_L3\_TCM (0x80000008) ) or the name 
of the generated tracefile (./trace/kmeans.Kmeans\_compss\_trace\_1460456106.prv). When using NIO communications adaptor with debug activated, the log of each worker
also contains the Extrae initialization information.

\subsubsection{Instrumented Threads}


Basic traces instrument the following threads:

\begin{itemize}
 \item Master node (3 threads)
 \begin {itemize}
 \item COMPSs runtime
 \item Task Dispatcher
 \item Access Processor
 \end{itemize}
 \item Worker node (1 + Computing Units)
 \begin{itemize}
  \item Main thread
  \item Number of threads available for computing
 \end{itemize}
\end{itemize}

\subsubsection{Information Available}

The basic mode tracefiles contain three kinds of information:

\begin{description}
 \item [Events,] marking diverse situations such as the runtime start, tasks' execution or synchronization points.
 \item [Communications,] showing the transfers and requests of the parameters needed by COMPSs tasks.
 \item [Hardware counters,] of the execution obtained with Performance API (see PAPI counters appendix \ref{sec:papi})
\end{description}


\subsubsection{Trace Example}

Figure \ref{fig:basic_trace} a tracefile generated by the execution of a k-means clustering algorithm. Each timeline contains information of a 
different resource, and each event's name is on the legend. Depending on the number of computing threads specified for each worker, the number of timelines varies. 
However the following threads are always shown:



\begin{description}
 \item [Master - Thread 1.1.1, ] \hfill \\ this timeline shows the actions performed by the main thread of the COMPSs application
 \item [Task Dispatcher - Thread 1.1.2,] \hfill \\ shows information about the state and scheduling of the tasks to be executed.
 \item [Access Processor - Thread 1.1.3,] \hfill \\  all the events related to the tasks' parameters management, such as dependencies or transfers are shown in this thread.
 \item [Worker X Master - Thread 1.X.1,] \hfill \\ this thread is the master of each worker and handles the computing resources and transfers.  Is is repeated for each available resource. All data events of the worker, such as requests, transfers and receives are marked on this timeline (when using the appropriate configurations).
 \item [Worker X Computing Unit Y - Thread 1.X.Y] \hfill \\ shows the actual tasks execution information and  is repeated as many times as computing threads has the worker X
\end{description}


\begin{landscape}
\begin{figure}[ht!]
  \centering
    \includegraphics[width=\linewidth]{./Sections/2_Execution/Figures/basic.png}
    \caption{Basic mode tracefile for a k-means algorithm visualized with compss\_runtime.cfg}
    \label{fig:basic_trace}
\end{figure}
\end{landscape}

\subsection{Advanced Mode}
This mode is for more advanced COMPSs' users and developers who want to customize further the information provided by the tracing or need rawer information like pthreads calls or Java garbage collection. With it, every single thread created during the execution is traced.
\\
\\
\textbf{N.B.:} The extra information provided by the advanced mode is only available on the workers when using NIO adaptor.


\subsubsection{Usage}

In order to activate the advanced tracing add the following option to the execution:

\begin{itemize}
 \item --tracing=advanced
\end{itemize}

\noindent Examples given:

\begin{lstlisting}[language=bash]
runcompss --tracing=advanced application_name application_args
\end{lstlisting}

\noindent Figure \ref{fig:advanced_trace} was generated as follows:


\begin{lstlisting}[language=bash]
runcompss \
     --lang=java \
     --tracing=advanced \
     --classpath=/path/to/jar/kmeans.jar \
     kmeans.KMeans
\end{lstlisting}


When advanced tracing is activated, the configuration file reported on the output is \textit{extrae\_advanced.xml}. 

\begin{lstlisting}[language=bash]
*** RUNNING JAVA APPLICATION KMEANS
...
...
...
Welcome to Extrae 3.3.0 (revision 3966 based on extrae/trunk)
Extrae: Parsing the configuration file (/opt/COMPSs/Runtime/scripts/user/../../configuration/xml/tracing/extrae_advanced.xml) begins
\end{lstlisting}

This is the default file used for advanced tracing. However, advanced users can modify it in order to customize the information provided by Extrae. The configuration file is read first by the master on the \textit{runcompss} script. When using NIO adaptor for communication, the configuration file is also read when each worker is started (on \textit{persistent\_worker.sh} or \textit{persistent\_worker\_starter.sh} depending on the execution environment).

If the default file is modified, the changes always affect the master, and also the workers when using NIO. Modifying the scripts which turn on the master and the workers is possible to achieve different instrumentations for master/workers. However, not all Extrae available XML configurations work with COMPSs,
some of them can make the runtime or workers crash so modify them at your discretion and risk. More information about instrumentation XML configurations
on Extrae User Guide at: 
\url{https://www.bsc.es/computer-sciences/performance-tools/trace-generation/extrae/extrae-user-guide}.


\subsubsection{Instrumented Threads}

Advanced mode instruments all the pthreads created during the application execution. It contains all the threads shown on basic traces plus extra ones used to
call command-line commands, I/O streams managers and all actions which create a new process. Due to the temporal nature of many of this threads, they may contain little information or appear just at specific parts of the execution pipeline.

\subsubsection{Information Available}

The advanced mode tracefiles contain the same information as the basic ones:


\begin{description}
 \item [Events,] marking diverse situations such as the runtime start, tasks' execution or synchronization points.
 \item [Communications,] showing the transfers and requests of the parameters needed by COMPSs tasks.
 \item [Hardware counters,] of the execution obtained with Performance API (see PAPI counters appendix \ref{sec:papi})
\end{description}


\subsubsection{Trace Example}

Figure \ref{fig:advanced_trace} shows the total completed instructions for a sample program executed with the advanced tracing mode. Note that the thread - resource correspondence described on the basic trace example is no longer static and thus cannot be inferred. Nonetheless, they can be found thanks to the named events shown in other configurations such as \textit{compss\_runtime.cfg}.


\begin{landscape}
\begin{figure}[ht!]
  \centering
    \includegraphics[width=\linewidth]{./Sections/2_Execution/Figures/advanced.png}
    \caption{Advanced mode tracefile for a testing program showing the total completed instructions}
    \label{fig:advanced_trace}
\end{figure}
\end{landscape}

For further information about Extrae, please visit the following site: 
\begin{center}
\url{http://www.bsc.es/computer-science/extrae} 
\end{center}
  
  \section{Visualization}
\label{sec:Visualization}

Paraver is the BSC tool for trace visualization. Trace events are encoded in Paraver format (.prv) by the Extrae tool. Paraver is a powerful
tool and allows users to show many views of the trace data using different configuration files. Users can manually load, edit or create
configuration files to obtain different tracing views. 

The following subsections explain how to load a trace file into Paraver, open the task events view using an already predefined 
configuration file, and how to adjust the view to display the data properly.

For further information about Paraver, please visit the following site:

\begin{center}
\url{http://www.bsc.es/computer-sciences/performance-tools/paraver}
\end{center}

\subsection{Trace Loading}
The final trace file in Paraver format (.prv) is at the base log folder of the application execution inside the trace folder. The fastest
way to open it is calling the Paraver binary directly using the tracefile name as the argument.

\begin{lstlisting}[language=bash]
wxparaver /path/to/trace/trace.prv
\end{lstlisting}
 
\subsection{Configurations}

To see the different events, counters and communications that the runtime generates, diverse configurations are available with the COMPSs
installation. To open one of them, go to the ``Load Configuration'' option in the main window and select ``File''. The configuration files
are under the following path for the default installation \verb|/opt/COMPSs/Dependencies/| \verb|paraver/cfgs/|. A detailed list of all the
available configurations can be found in Appendix \ref{sec:configs}.

The following guide uses the \textit{compss\_tasks.cfg} as an example to illustrate the basic usage of Paraver.
After accepting the load of the configuration file, another window appears showing the view. Figures \ref{fig:trace_1} and
\ref{fig:trace_2} show an example of this process.

\begin{figure}[ht!]
  \centering
    \includegraphics[width=0.45\textwidth]{./Sections/3_Visualization/Figures/1.jpeg}
    \caption{Paraver menu}
    \label{fig:trace_1}
\end{figure}

\begin{figure}[ht!]
  \centering
    \includegraphics[width=1.0\textwidth]{./Sections/3_Visualization/Figures/2.jpeg}
    \caption{Trace file}
    \label{fig:trace_2}
\end{figure}

\subsection{View Adjustment}

In a Paraver view, a red exclamation sign may appear in the bottom-left corner (see Figure \ref{fig:trace_2} in the previous section). This
means that some event values are not being shown (because they are out of the current view scope), so little adjustments must be made to
view the trace correctly:

\begin{itemize}
 \item Fit window: modifies the view scope to fit and display all the events in the current window.
	\begin{itemize}
	    \item Right click on the trace window
	    \item Choose the option Fit Semantic Scale / Fit Both
	\end{itemize}
\end{itemize}

\begin{figure}[ht!]
  \centering
    \includegraphics[width=1.0\textwidth]{./Sections/3_Visualization/Figures/3.jpeg}
    \caption{Paraver view adjustment: Fit window}
\end{figure}

\begin{itemize} 
 \item View Event Flags: marks with a green flag all the emitted the events.
	\begin{itemize}
	    \item Right click on the trace window
	    \item Chose the option View / Event Flags
	\end{itemize}
\end{itemize}
 
\begin{figure}[ht!]
  \centering
    \includegraphics[width=1.0\textwidth]{./Sections/3_Visualization/Figures/4.jpeg}
    \caption{Paraver view adjustment: View Event Flags}
\end{figure}

\begin{itemize}
 \item Show Info Panel: display the information panel. In the tab ``Colors'' we can see the legend of the colors shown in the view.
	\begin{itemize}
	    \item Right click on the trace window
	    \item Check the Info Panel option
	    \item Select the Colors tab in the panel
	\end{itemize}
\end{itemize}

\begin{figure}[ht!]
  \centering
    \includegraphics[width=1.0\textwidth]{./Sections/3_Visualization/Figures/5.jpeg}
    \caption{Paraver view adjustment: Show info panel}
\end{figure}

\begin{itemize}
 \item Zoom: explore the tracefile more in-depth by zooming into the most relevant sections.
	\begin{itemize}
	    \item Select a region in the trace window to see that region in detail
	    \item Repeat the previous step as many times as needed
	    \item The undo-zoom option is in the right click panel
	\end{itemize}
\end{itemize}

\begin{figure}[ht!]
  \centering
    \includegraphics[width=1.0\textwidth]{./Sections/3_Visualization/Figures/6.jpeg}
    \caption{Paraver view adjustment: Zoom configuration}
\end{figure}

\begin{figure}[ht!]
  \centering
    \includegraphics[width=1.0\textwidth]{./Sections/3_Visualization/Figures/6_2.jpeg}
    \caption{Paraver view adjustment: Zoom configuration}
\end{figure}

  
  \section{Interpretation}
\label{sec:Interpretation}

This section explains how to interpret a trace view once it has been adjusted as 
described in the previous section.

\begin{itemize}
 \item The trace view has on its horizontal axis the execution time and on the vertical 
       axis one line for the master at the top, and below it, one line for each of the workers.
 \item In a line, the light blue color is associated with an idle state, i.e. there is no event at that time.
 \item Whenever an event starts or ends a flag is shown.
 \item In the middle of an event, the line shows a different color. Colors are assigned depending on the event type.
 \item The info panel contains the legend of the assigned colors to each event type.
\end{itemize}

\begin{figure}[ht!]
  \centering
    \includegraphics[width=\textwidth]{./Sections/4_Interpretation/Figures/7.jpeg}
    \caption{Trace interpretation}
\end{figure}

  \section{Analysis}
\label{sec:Analysis}
In this section, we will give some tips to analyse a COMPSs trace from two different points of view:
graphically and numerically.

\subsection{Graphical Analysis}
The main concept is that computational events, the task events in this case, must be well 
distributed among all workers to have a good parallelism, and the duration of task events 
should be also balanced, this means, the duration of computational bursts.

\begin{figure}[ht!]
  \centering
    \includegraphics[width=1.0\textwidth]{./Sections/5_Analysis/Figures/8.jpeg}
    \caption{Caption.}
\end{figure}

In the previous trace view, all the tasks of type ``hmmpfam'' in dark blue appear to be well 
distributed among the four workers, each worker executes four ``hmmpfam'' tasks.

But some workers finish earlier than the others, worker 1.2.3 finish the first and worker 1.2.1 
the last. So there is an imbalance in the duration of ``hmmpfam'' tasks. The programmer should 
analyse then whether all the tasks process the same amount of input data and do the same thing 
in order to find out the reason of such imbalance.

Another thing to highlight is that tasks of type ``scoreRatingSameDB'' are not equal distributed 
among all the workers. There are workers that execute more tasks of this type than the others. 
To understand better what happens here, let’s take a look to the execution graph and also zoom 
in the last part of the trace.

\begin{figure}[ht!]
  \centering
    \includegraphics[width=1.0\textwidth]{./Sections/5_Analysis/Figures/9.jpeg}
    \caption{Caption.}
\end{figure}

\begin{figure}[ht!]
  \centering
    \includegraphics[width=1.0\textwidth]{./Sections/5_Analysis/Figures/10.jpeg}
    \caption{Caption.}
\end{figure}

There is only one task of type ``scoreRatingSameSeq''. This task appears in red in the trace 
(and in light-green in the graph). With the help of the graph we see that the ``scoreRatingSameSeq'' 
task has dependences on tasks of type ``scoreRatingSameDB'', in white (or yellow).

When the last task of type ``hmmpfam'' (in dark blue) ends, the last dependences are solved, 
and if we look at the graph, this means going across a path of three dependences of type 
``scoreRatingSameDB'' (in yellow). And because of these are sequential dependences (one depends 
on the previous) no more than a worker can be used at the same time to execute the tasks. 
This is the reason of why the last three task of type ``scoreRatingSameDB'' (in white) are 
executed in worker 1.2.1 sequentially.

\subsection{Numerical Analysis}
Here we show another trace from a different parallel execution of the Hmmer program.
 
\begin{figure}[ht!]
  \centering
    \includegraphics[width=1.0\textwidth]{./Sections/5_Analysis/Figures/11.jpeg}
    \caption{Caption.}
\end{figure} 
 
Paraver offers the possibility of having different histograms of the trace events. 
For it just click the ``New Histogram'' button in the main window and accept the 
default options in the ``New Histogram'' window that will appear.

\begin{figure}[ht!]
  \centering
    \includegraphics[width=0.5\textwidth]{./Sections/5_Analysis/Figures/12.jpeg}
    \caption{Paraver Menu - New Histogram}
\end{figure}

After that, the following table is shown. In this case for each worker, the time spent 
executing each type of task is shown. Task names appear in the same color than in the 
trace view. The color of a cell in a row corresponding to a worker goes in a scale from 
a light-green for lower values to a dark-blue for higher ones. This conforms a color based histogram.

\begin{figure}[ht!]
  \centering
    \includegraphics[width=0.8\textwidth]{./Sections/5_Analysis/Figures/13.jpeg}
    \caption{Hmmpfam histogram}
\end{figure}
 
The previous table also gives, at the end of each column, some extra statistical 
information for each type of tasks (as the total, average, maximum or minimum values, etc.).

\newpage
In the window properties of the main window we can change the semantic of the statistics 
to see other factors rather than the time, for example, the number of bursts.

\begin{figure}[ht!]
  \centering
    \includegraphics[width=0.8\textwidth]{./Sections/5_Analysis/Figures/14.jpeg}
    \caption{Paraver histogram options menu}
\end{figure}

\newpage
In the same way as before, the following table shows for each worker the number of bursts 
for each type of task, this is, the number or tasks executed of each type. Notice the gradient 
scale from light-green to dark-blue changes with the new values.

\begin{figure}[ht!]
  \centering
    \includegraphics[width=0.8\textwidth]{./Sections/5_Analysis/Figures/15.jpeg}
    \caption{Hmmpfam histogram with the number of bursts}
\end{figure}
  
  \newpage

  \begin{appendices}
  
  \section{PAPI: Hardware Counters}
\label{sec:papi}

The applications instrumentation supports hardware counters through the performance API (PAPI). In order to use it, PAPI needs to be present on the machine before installing
COMPSs. 

During COMPSs installation it is possible to check if PAPI has been detected in the Extrae config report:

\begin{lstlisting}[language=bash]
Package configuration for Extrae VERSION based on extrae/trunk rev. XXXX:
-----------------------
Installation prefix: /opt/COMPSs/Dependencies/extrae
Cross compilation: no
...
...
...

Performance counters: yes
  Performance API: PAPI
  PAPI home: /usr
  Sampling support: yes
\end{lstlisting}

\textbf{N.B.} PAPI detection is only performed in the machine where COMPSs is installed. User is responsible of providing a valid PAPI installation to the worker machines to be used (if they are different from the master), otherwise workers will crash because of the missing \textit{libpapi.so}. 


PAPI installation and requirements depend on the OS. On Ubuntu 14.04 it is available under textit{papi-tools} package; on OpenSuse textit{papi} and textit{papi-dev}.
For more information check \url{https://icl.cs.utk.edu/projects/papi/wiki/Installing_PAPI}.


Extrae only supports 8 active hardware counters at the same time. Both basic and advanced mode have the same default counters list:

\begin{description}
 \item [PAPI\_TOT\_INS] Instructions completed
 \item [PAPI\_TOT\_CYC] Total cycles
 \item [PAPI\_LD\_INS] Load instructions
 \item [PAPI\_SR\_INS] Store instructions
 \item [PAPI\_BR\_UCN] Unconditional branch instructions
 \item [PAPI\_BR\_CN] Conditional branch instructions
 \item [PAPI\_VEC\_SP] Single precision vector/SIMD instructions
 \item [RESOURCE\_STALLS] Cycles Allocation is stalled due to Resource Related reason
\end{description}

The XML config file contains a secondary set of counters. In order to activate it just change the \textit{starting-set-distribution} from 2 to 1 under the \textit{cpu} tag. The second set provides the following information:

\begin{description}
 \item [PAPI\_TOT\_INS] Instructions completed
 \item [PAPI\_TOT\_CYC] Total cycles
 \item [PAPI\_L1\_DCM] Level 1 data cache misses
 \item [PAPI\_L2\_DCM] Level 2 data cache misses
 \item [PAPI\_L3\_TCM] Level 3 cache misses
 \item [PAPI\_FP\_INS] Floating point instructions
\end{description}

To further customize the tracked counters, modify the XML to suit your needs. To find the available PAPI counters on a given computer
issue the command \textit{papi\_avail -a}. For more information about Extrae's XML configuration refer to
\url{https://www.bsc.es/computer-sciences/performance-tools/trace-generation/extrae/extrae-user-guide}.


\section{Paraver: configurations}
\label{sec:configs}

Tables \ref{tab:paraver_configs_general}, \ref{tab:paraver_configs_python} and \ref{tab:paraver_configs_comm} provide information 
about the different pre-build configurations that are distributed with COMPSs and that can be found under 
the \verb|/opt/COMPSs/Dependencies/| \verb|paraver/cfgs/| folder. The \textit{cfgs} folder contains all the basic views, the \textit{python}
folder contains the configurations for Python events, and finally the \textit{comm} folder contains the configurations related 
to communications.

~ \newline

\bgroup
  \def\arraystretch{1.5}
  \begin{table}[!ht]
    \begin{center}
      \begin{tabular}{| p{0.45\textwidth} | p{0.45\textwidth} |}
	\hline
	$2dp\_runtime\_state.cfg$		& 2D plot of runtime state \\ \hline
	$2dp\_tasks.cfg$			& 2D plot of tasks duration \\ \hline
	$3dh\_duration\_runtime.cfg$		& 3D Histogram of runtime execution \\ \hline
	$3dh\_duration\_tasks.cfg$		& 3D Histogram of tasks duration \\ \hline
	$compss\_runtime.cfg$ 			& Shows COMPSs Runtime events (master and workers) \\ \hline
	$compss\_tasks\_and\_runtime.cfg$ 	& Shows COMPSs Runtime events (master and workers) and tasks execution \\ \hline
	$compss\_tasks.cfg$ 			& Shows tasks execution \\ \hline
	$compss\_tasks\_numbers.cfg$ 		& Shows tasks execution by task id \\ \hline
	$Interval\_between\_runtime.cfg$ 	& Interval between runtime events \\ \hline
	$thread\_cpu.cfg$			& Shows the initial executing CPU. \\ \hline
      \end{tabular}
      \caption{General paraver configurations for COMPSs Applications}
      \label{tab:paraver_configs_general}
    \end{center}
  \end{table}
\egroup

\bgroup
  \def\arraystretch{1.5}
  \begin{table}[!ht]
    \begin{center}
      \begin{tabular}{| p{0.45\textwidth} | p{0.45\textwidth} |}
	\hline
	$3dh\_events\_inside\_task.cfg$		& 3D Histogram of python events \\ \hline
	$3dh\_events\_inside\_tasks.cfg$	& Events showing python information such as user function execution time, modules imports, or serializations. \\ \hline
      \end{tabular}
      \caption{Available paraver configurations for Python events of COMPSs Applications}
      \label{tab:paraver_configs_python}
    \end{center}
  \end{table}
\egroup

\newpage

\bgroup
  \def\arraystretch{1.5}
  \begin{table}[!ht]
    \begin{center}
      \begin{tabular}{| p{0.45\textwidth} | p{0.45\textwidth} |}
	\hline
	$sr\_bandwith.cfg$ 			& Send/Receive bandwith view for each node \\ \hline
	$send\_bandwith.cfg$ 			& Send bandwith view for each node \\ \hline
	$receive\_bandwith.cfg$ 		& Receive bandwith view for each node \\ \hline
	$process\_bandwith.cfg$ 		& Send/Receive bandwith table for each node \\ \hline
	$compss\_tasks\_scheduling\_transfers.cfg$ 		& Task's transfers requests for scheduling (gradient of tasks ID) \\ \hline
	$compss\_tasksID\_transfers.cfg$ 	& Task's transfers request for each task (task with its IDs are also shown) \\ \hline
	$compss\_data\_transfers.cfg$ 		& Shows data transfers for each task's parameter \\ \hline
	$communication\_matrix.cfg$ 		& Table view of communications between each node \\ \hline
      \end{tabular}
      \caption{Available paraver configurations for COMPSs Applications}
      \label{tab:paraver_configs_comm}
    \end{center}
  \end{table}
\egroup


\section{User Events in Python}

Users can emit custom events inside their python \textbf{tasks}. Thanks to the fact that python isn't a compiled language, 
users can emit events inside their own tasks using the available extrae instrumentation object because it is already imported.
~ \newline

To emit an event first \verb|import pyextrae| just use the call \verb|pyextrae.event(type, id)| or \verb|pyextrae.eventand| \verb|counters (type, id)|
if you also want to emit PAPI hardware counters. It is recommended to use a type number higher than 8000050 in order 
to avoid type's conflicts. This events will appear automatically on the generated trace. In order to visualize them, take, 
for example, \verb|compss_runtime.cfg| and go to \verb|Window Properties -> Filter -> Events| \verb|-> Event Type| and change the 
value labeled \textit{Types} for your custom events type. If you want to name the events, you will need to manually add them to the .pcf file. Paraver uses by default the .pcf with the same name as the tracefile so if you add them to one, you can reuse it just by changing its name to the tracefile.q
~ \newline

More information and examples of common python usage can be found under the default directory 
\verb|/opt/COMPSs/Dependencies/extrae/share/examples/PYTHON|.

  
  \end{appendices}
  
  %%%%%%%%%%%%% END PAGE %%%%%%%%%%%%%%
  \newpage

  \vspace*{\fill} 
  \begin{center}
    \large { Please find more details on the COMPSs framework at }
    \huge{\url{http://compss.bsc.es}}
  \end{center}    
  \vspace*{\fill} 
           
\end{document}

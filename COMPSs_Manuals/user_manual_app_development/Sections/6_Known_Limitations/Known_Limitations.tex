\section{Known Limitations}
\label{sec:Known_Limitations}

The current COMPSs version (\compssversion) has the following limitations: 
\begin{itemize}
 \item \textbf{Exceptions:} \newline The current COMPSs version is not able to propagate exceptions raised from a task to the master. However, the runtime
 catches any exception and sets the task as failed.
 
 \item \textbf{Java tasks:} \newline Java tasks \textbf{must} be declared as \textbf{public}. Despite the fact that tasks can be
 defined in the main class or in other ones, we recommend to define the tasks in a separated class from the main method to force
 its public declaration.
 
 \item \textbf{Java objects:} \newline Objects used by tasks must follow the \textit{java beans} model (implementing an empty 
 constructor and getters and setters for each attribute) or implement the \textit{serializable} interface. This is due to the 
 fact that objects will be transferred to remote machines to execute the tasks.
 
 \item \textbf{Java object aliasing:} \newline If a task has an object parameter and returns an object, the returned value
 must be a new object (or a cloned one) to prevent any aliasing with the task parameters. 
 \begin{lstlisting}[language=java]
// @Method(declaringClass = "...")
// DummyObject incorrectTask (
//    @Parameter(type = Type.OBJECT, direction = Direction.IN) DummyObject a,
//    @Parameter(type = Type.OBJECT, direction = Direction.IN) DummyObject b
// );
public DummyObject incorrectTask (DummyObject a, DummyObject b) {
    if (a.getValue() > b.getValue()) {
        return a;
    }
    return b;
}

// @Method(declaringClass = "...")
// DummyObject correctTask (
//    @Parameter(type = Type.OBJECT, direction = Direction.IN) DummyObject a,
//    @Parameter(type = Type.OBJECT, direction = Direction.IN) DummyObject b
// );
public DummyObject correctTask (DummyObject a, DummyObject b) {
    if (a.getValue() > b.getValue()) {
        return a.clone();
    }
    return b.clone();
}

public static void main() {
    DummyObject a1 = new DummyObject();
    DummyObject b1 = new DummyObject();
    DummyObject c1 = new DummyObject();
    c1 = incorrectTask(a1, b1);
    System.out.println("Initial value: " + c1.getValue());
    a1.modify();
    b1.modify();
    System.out.println("Aliased value: " + c1.getValue());
    
    
    DummyObject a2 = new DummyObject();
    DummyObject b2 = new DummyObject();
    DummyObject c2 = new DummyObject();
    c2 = incorrectTask(a2, b2);
    System.out.println("Initial value: " + c2.getValue());
    a2.modify();
    b2.modify();
    System.out.println("Non-aliased value: " + c2.getValue());
}
\end{lstlisting}
 
 \item \textbf{Services types:} \newline The current COMPSs version only supports SOAP based services that implement the WS 
 interoperability standard. REST services are not supported.
 
 \item \textbf{Use of file paths:} \newline The persistent workers implementation has a unique \textit{Working Directory} per 
 worker. That means that tasks should not use hardcoded file names to avoid file collisions and tasks misbehaviours. We recommend to
 use files declared as task parameters, or to manually create a sandbox inside each task execution and/or to generate temporary 
 random file names. 
 
 \item \textbf{Python constraints in the cloud:} \newline When using python applications with constraints in the cloud the minimum
 number of VMs must be set to 0 because the initial VM creation doesn't respect the tasks contraints. Notice that if no contraints are
 defined the initial VMs are still usable. 
 
 \item \textbf{Intermediate files}: \newline Some applications may generate intermediate files that are only used among tasks
 and are never needed inside the master's code. However, COMPSs will transfer back these files to the master node at the end of the 
 execution. Currently, the only way to avoid transferring these intermediate files is to manually erase them at the end of the
 master's code. Users must take into account that this only applies for files declared as task parameters and \textbf{not} for files
 created and/or erased inside a task. 

\item \textbf{Python object hierarchy dependency detection}: \newline Dependencies are detected only on the objects that are task 
parameters or outputs. Consider the following code:
\begin{lstlisting}[language=python]
# a.py
class A:
  def __init__(self, b):
    self.b  = b
# main.py
from a import A
from pycompss.api.task import task
from pycompss.api.parameter import *

@task(obj = IN, returns = int)
def get_b(obj):
  return obj.b

@task(obj = INOUT)
def inc(obj):
  obj += [1]

def main():
  from pycompss.api.api import compss_wait_on
  my_a = A([5])
  inc(my_a.b)
  obj = get_b(my_a)
  obj = compss_wait_on(obj)
  print obj

if __name__ == '__main__':
  main()            
\end{lstlisting}
Note that there should exist a dependency between \verb|A| and \verb|A.b|. However, PyCOMPSs is not capable
to detect dependencies of that kind. These dependencies
must be handled (and avoided) manually.

\item \textbf{Python modules with global states}:\newline Some modules (for example \verb|logging|) have internal variables apart from functions. 
These modules are not guaranteed to work in PyCOMPSs due to the fact that master and worker code are executed in different interpreters. For instance, if a \verb|logging| configuration is set on some
worker, it will not be visible from the master interpreter instance.

\item \textbf{Python global variables}:\newline
This issue is very similar to the previous one. PyCOMPSs does not guarantee that applications that create or modify global variables while
worker code is executed will work. In particular, this issue (and the previous one) is due to Python's Global Interpreter Lock (GIL).

\item \textbf{Python application directory as a module}:\newline
If the Python application root folder is a python module (i.e: it contains an \verb|__init__.py| file) then \verb|runcompss| must be called from
the parent folder. For example, if the Python application is in a folder with an \verb|__init__.py| file named \verb|my_folder| then PyCOMPSs will
resolve all functions, classes and variables as \verb|my_folder.object_name| instead of \verb|object_name|. For example, consider the following file
tree:
\begin{lstlisting}[]
my_apps/
|- kmeans/
    |- __init__.py
    |- kmeans.py
\end{lstlisting}
Then the correct command to call this app is \verb|runcompss kmeans/kmeans.py| from the \verb|my_apps| directory.

\item \textbf{Python early program exit}:\newline
All intentional, premature exit operations must be done with \verb|sys.exit|. PyCOMPSs needs to perform some cleanup tasks before exiting and, if an early exit is performed with \verb|sys.exit|, the event will be captured,
allowing PyCOMPSs to perform these tasks. If the exit operation is done in a different way then there is no guarantee that the application will end properly.

\item \textbf{Python with numpy and MKL}:\newline
Tasks that invoke numpy and MKL may experience issues if tasks use a different number of MKL threads. This is due to the fact that MKL reuses 
threads along different calls and it does not change the number of threads from one call to another.


\end{itemize}


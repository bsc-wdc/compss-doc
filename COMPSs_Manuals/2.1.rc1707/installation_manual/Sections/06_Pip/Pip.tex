\section{Pip}
\label{sec:Pip}

\subsection{Pre-requisites}
\label{subsec:pip_prerequisites}
In order to be able to install COMPSs and PyCOMPSs with Pip the following requirements must be met:
\begin{enumerate}
 \item Have all the dependencies (excluding the COMPSs packages) mentioned in the section \ref{subsec:packages_dependencies} satisfied and Python pip.
 As an example for some distributions:
   \subitem {\bf Fedora 25} dependencies installation command:
     \begin{lstlisting}[language=bash]
       sudo dnf install -y java-1.8.0-openjdk java-1.8.0-openjdk-devel graphviz xdg-utils libtool automake python python-libs python-pip python-devel python2-decorator boost-devel boost-serialization boost-iostreams libxml2 libxml2-devel gcc gcc-c++ gcc-gfortran tcsh @development-tools redhat-rpm-config papi
       # If the libxml softlink is not created during the installation of libxml2, the COMPSs
       # installation may fail.
       # In this case, that softlink has to be created manually with the following command:
       sudo ln -s /usr/include/libxml2/libxml/ /usr/include/libxml
     \end{lstlisting}
   \subitem {\bf Ubuntu 16.04} dependencies installation command:
     \begin{lstlisting}
       sudo apt-get install -y openjdk-8-jdk graphviz xdg-utils libtool automake build-essential python2.7 libpython2.7 libboost-serialization-dev libboost-iostreams-dev  libxml2 libxml2-dev csh gfortran python-pip libpapi-dev
     \end{lstlisting}
   \subitem {\bf OpenSuse 42.2} dependencies installation command:
     \begin{lstlisting}
       sudo zypper install --type pattern -y devel_basis
       sudo zypper install -y java-1_8_0-openjdk-headless java-1_8_0-openjdk java-1_8_0-openjdk-devel graphviz xdg-utils python python-devel libpython2_7-1_0 python-decorator libtool automake  boost-devel libboost_serialization1_54_0 libboost_iostreams1_54_0  libxml2-2 libxml2-devel tcsh gcc-fortran python-pip papi libpapi
     \end{lstlisting}
   \subitem {\bf Debian 8} dependencies installation command:
     \begin{lstlisting}
        su -
        echo "deb http://ppa.launchpad.net/webupd8team/java/ubuntu xenial main" | tee /etc/apt/sources.list.d/webupd8team-java.list
        echo "deb-src http://ppa.launchpad.net/webupd8team/java/ubuntu xenial main" | tee -a /etc/apt/sources.list.d/webupd8team-java.list
        apt-key adv --keyserver hkp://keyserver.ubuntu.com:80 --recv-keys EEA14886
        apt-get update
        apt-get install oracle-java8-installer
        apt-get install graphviz xdg-utils libtool automake build-essential python python-decorator python-pip python-dev libboost-serialization1.55.0 libboost-iostreams1.55.0 libxml2 libxml2-dev libboost-dev csh gfortran papi-tools
     \end{lstlisting}
   \subitem {\bf CentOS 7} dependencies installation command:
     \begin{lstlisting}
        sudo rpm -iUvh https://dl.fedoraproject.org/pub/epel/epel-release-latest-7.noarch.rpm
        sudo yum -y update
        sudo yum install java-1.8.0-openjdk java-1.8.0-openjdk-devel graphviz xdg-utils libtool automake python python-libs python-pip python-devel python2-decorator boost-devel boost-serialization boost-iostreams libxml2 libxml2-devel gcc gcc-c++ gcc-gfortran tcsh @development-tools redhat-rpm-config papi
        sudo pip install decorator
     \end{lstlisting}
 \item Have a proper \verb|JAVA_HOME| environment variable definition. This variable must contain a valid path to a Java JDK (as a remark, it must point to a JDK, not JRE). A possible value is the following:
 \begin{lstlisting}[language=bash]
  user@machine:~> echo $JAVA_HOME
  /usr/lib64/jvm/java-openjdk/\end{lstlisting}
\end{enumerate}


\subsection{Installation}
\label{subsec:pip_installation}
Depending on the machine, the installation command may vary. It must be assured that the Python2.7 pip is being executed. Some of the possible scenarios and their proper installation command are:
\begin{enumerate}
 \item There is more than one Python version installed:
 \begin{lstlisting}[language=bash]
 sudo -E python2.7 -m pip install pycompss -v\end{lstlisting}
 \item There is only Python2.7 installed:
 \begin{lstlisting}[language=bash]
 sudo -E pip install pycompss -v \end{lstlisting}
\end{enumerate}
It is recommended to restart the user session once the installation process has finished.

Alternatively, the following command sets all the COMPSs environment.
\begin{lstlisting}[language=bash]
source /etc/profile.d/compss.sh
\end{lstlisting}
However, this command should be executed in every different terminal during the current user session.


\subsection{Configuration}
\label{subsec:pip_configuration}
The steps mentioned in Section \ref{subsec:Passwordless_ssh} must be done in order to have a functional COMPSs and pyCOMPSs installation.


\subsection{Post installation}
As mentioned in Section \ref{subsec:pip_installation}, it is recommended to restart the user session once the installation process has finished.

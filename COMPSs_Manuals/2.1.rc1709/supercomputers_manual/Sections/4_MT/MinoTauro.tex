\section{MinoTauro}
\label{sec:minotauro}

\subsection{Basic queue commands}

The MinoTauro supercomputer uses the SLURM (Simple Linux Utility for Resource Management) workload manager. The basic commands 
to manage jobs are listed below:

\begin{itemize}
 \item \textbf{sbatch} Submit a batch job to the SLURM system
 \item \textbf{scancel} Kill a running job 
 \item \textbf{squeue -u $<$username$>$} See the status of jobs in the SLURM queue
\end{itemize}

For more extended information please check the \textit{SLURM: Quick start user guide} at 
\url{https://slurm.schedmd.com/quickstart.html} .


\subsection{Tracking COMPSs jobs}

When submitting a COMPSs job a temporal file will be created storing the job information. For example:

\begin{lstlisting}[language=bash]
$ enqueue_compss \
  --exec_time=15 \
  --num_nodes=3 \
  --cpus_per_node=16 \
  --master_working_dir=. \
  --worker_working_dir=gpfs \
  --lang=python \
  --log_level=debug \
  <APP> <APP_PARAMETERS>

  
SC Configuration:          default.cfg
Queue:                     default
Reservation:               disabled
Num Nodes:                 3
Num Switches:              0
GPUs per node:             0
Job dependency:            None
Exec-Time:                 00:15
Storage Home:              null
Storage Properties:        null
Other:                     
        --sc_cfg=default.cfg
        --cpus_per_node=16
        --master_working_dir=.
        --worker_working_dir=gpfs
        --lang=python
        --classpath=.
        --library_path=.
        --comm=integratedtoolkit.nio.master.NIOAdaptor
        --tracing=false
        --graph=false
        --pythonpath=.
        <APP> <APP_PARAMETERS>
Temp submit script is: /scratch/tmp/tmp.pBG5yfFxEo

$ cat /scratch/tmp/tmp.pBG5yfFxEo
#!/bin/bash
#
#SBATCH --job-name=COMPSs
#SBATCH --workdir=. 
#SBATCH -o compss-%J.out
#SBATCH -e compss-%J.err
#SBATCH -N 3
#SBATCH -n 48
#SBATCH --exclusive
#SBATCH -t00:15:00 
...
\end{lstlisting}

In order to trac the jobs state users can run the following command:
\begin{lstlisting}[language=bash]
$ squeue
JOBID  PARTITION   NAME    USER  ST  TIME    NODES  NODELIST (REASON)
XXXX   projects    COMPSs   XX   R   00:02       3  nvb[6-8]   
\end{lstlisting}

The specific COMPSs logs are stored under the \verb|~/.COMPSs/| folder; saved as a local \textit{runcompss} execution. For further 
details please check \textit{COMPSs User Manual: Application Execution} available at our webpage \url{http://compss.bsc.es} .

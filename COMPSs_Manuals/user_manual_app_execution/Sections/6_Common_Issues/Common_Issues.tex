\section{Common Issues}
\label{sec:Common_Issues}

This section provides answers for the most common issues of the execution of COMPSs applications.
For specific issues not covered in this section, please do not hesitate to contact us at:

\begin{center}
  \textbf{\url{support-compss@bsc.es}}
\end{center}


%
% SUBSECTION
%
\subsection{How to debug}

When the application does not behave as expected the first thing users must do is to run it in \textbf{debug} mode executing
the \texttt{runcompss} command withthe \texttt{-d} flag to enable the debug log level.

In this case the application execution will produce the following files:

\begin{itemize}
 \item \texttt{runtime.log}
 \item \texttt{resources.log}
 \item \texttt{jobs} folder
\end{itemize}

First, users should check the last lines of the runtime.log. If the file-transfers or the tasks are failing an error message 
will appear in this file. If the file-transfers are successfully and the jobs are submitted, users should check the \texttt{jobs} folder and look 
at the error messages produced inside each job. Users should notice that if there are $\_RESUBMITTED$ files something 
inside the job is failing.


%
% SUBSECTION
%
\subsection{Tasks are not executed}

If the tasks remain in \textbf{Blocked} state probably there are no existing resources matching the specific task constraints. 
This error can be potentially caused by two facts: the resources are not correctly loaded into the runtime, or the task constraints do not match with any resource. 

In the first case, users should take a look at the \texttt{resouces.log} and check that all the resources
defined in the \texttt{project.xml} file are available to the runtime. In the second case users should re-define the task 
constraints taking into account the resources capabilities defined into the \texttt{resources.xml} and \texttt{project.xml} files.


%
% SUBSECTION
%
\subsection{Jobs fail}

If all the application's tasks fail because all the submitted jobs fail, it is probably due to the fact that there is a resource 
miss-configuration. In most of the cases, the resource that the application is trying to access has no passwordless access through
the configured user. This can be checked by:

\begin{itemize}
 \item Open the \texttt{project.xml}. (The default file is stored under \\ \texttt{/opt/COMPSs/
 Runtime/configuration/xml/projects/project.xml})
 \item For each resource annotate its name and the value inside the \texttt{User} tag. Remember that if there is no \texttt{User}
 tag COMPSs will try to connect this resource with the same username than the one that launches the main application.
 \item For each annotated resourceName - user please try \texttt{ssh user@resourceName}. If the connection asks for a password then
 there is an error in the configuration of the ssh access in the resource.
\end{itemize}

The problem can be solved running the following commands:

\begin{lstlisting}[language=bash]
compss@bsc:~$ scp ~/.ssh/id_dsa.pub user@resourceName:./mydsa.pub
compss@bsc:~$ ssh user@resourceName "cat mydsa.pub >> ~/.ssh/authorized_keys; rm ./mydsa.pub"
\end{lstlisting}

These commands are a quick solution, for further details please check the \textit{Additional Configuration} section 
inside the \textit{COMPSs Installation Manual} available at our website \url{http://compss.bsc.es}.


%
% SUBSECTION
%
\subsection{Exceptions when starting the Worker processes}

When the COMPSs master is not able to communicate with one of the COMPSs workers described in the \textit{project.xml} and 
\textit{resources.xml} files, different exceptions can be raised and logged on the \textit{runtime.log} of the application. All of
them are raised during the worker start up and contain the \texttt{[WorkerStarter]} prefix. Next we provide a list with the common
exceptions:

\begin{itemize}
 \item \texttt{InitNodeException}: Exception raised when the remote SSH process to start the worker has failed.
 \item \texttt{UnstartedNodeException}: Exception raised when the worker process has aborted.
 \item \texttt{Connection refused}: Exception raised when the master cannot communicate with the worker process (NIO).
\end{itemize}

All these exceptions encapsulate an error when starting the worker process. This means that \textbf{the worker machine is not properly configured} and thus,
you need to check the environment of the failing worker. Further information about the specific error can be found on the worker log,
available at the working directory path in the remote worker machine (the worker working directory specified in the \textit{project.xml}
file). 

Next, we list the most common errors and their solutions:

\begin{itemize}
 \item \textbf{java command not found}: Invalid path to the java binary. Check the \texttt{JAVA\_HOME} definition at the remote worker machine. 
 \item \textbf{Cannot create WD}: Invalid working directory. Check the rw permissions of the worker's working directory.
 \item \textbf{No exception}: The worker process has started normally and there is no exception. In this case the issue is normally due to the
 firewall configuration preventing the communication between the COMPSs master and worker. Please check that the worker firewall has in and out
 permissions for TCP and UDP in the adaptor ports (the adaptor ports are specified in the \textit{resources.xml} file. By default the port rank
 is 43000-44000.
\end{itemize}


%
% SUBSECTION
%
\subsection{Compilation error: @Method not found}

When trying to compile Java applications users can get some of the following compilation errors:
\begin{lstlisting}[language=bash]
error: package es.bsc.compss.types.annotations does not exist
import es.bsc.compss.types.annotations.Constraints;
                                          ^
error: package es.bsc.compss.types.annotations.task does not exist
import es.bsc.compss.types.annotations.task.Method;
                                          ^
error: package es.bsc.compss.types.annotations does not exist
import es.bsc.compss.types.annotations.Parameter;
                                          ^
error: package es.bsc.compss.types.annotations.Parameter does not exist
import es.bsc.compss.types.annotations.parameter.Direction;
                                                    ^
error: package es.bsc.compss.types.annotations.Parameter does not exist
import es.bsc.compss.types.annotations.parameter.Type;
                                                    ^
error: cannot find symbol
@Parameter(type = Type.FILE, direction = Direction.INOUT)
^
  symbol:   class Parameter
  location: interface APPLICATION_Itf
  
error: cannot find symbol
@Constraints(computingUnits = "2")
^
  symbol:   class Constraints
  location: interface APPLICATION_Itf
  
error: cannot find symbol
@Method(declaringClass = "application.ApplicationImpl")
^
  symbol:   class Method
  location: interface APPLICATION_Itf
\end{lstlisting}

All these errors are raised because the \texttt{compss-engine.jar} is not listed in the CLASSPATH. The default COMPSs installation
automatically inserts this package into the CLASSPATH but it may have been overwritten or deleted. Please check that your 
environment variable CLASSPATH containts the \texttt{compss-engine.jar} location by running the following command:

\begin{lstlisting}[language=bash]
$ echo $CLASSPATH | grep compss-engine
\end{lstlisting}

If the result of the previous command is empty it means that you are missing the \texttt{compss-engine.jar} package in your classpath. 

The easiest solution is to manually export the CLASSPATH variable into the user session:

\begin{lstlisting}[language=bash]
$ export CLASSPATH=$CLASSPATH:/opt/COMPSs/Runtime/compss-engine.jar
\end{lstlisting}

However, you will need to remember to export this variable every time you log out and back in again. Consequently, we recommend to 
add this export to the \texttt{.bashrc} file:

\begin{lstlisting}[language=bash]
$ echo "# COMPSs variables for Java compilation" >> ~/.bashrc
$ echo "export CLASSPATH=$CLASSPATH:/opt/COMPSs/Runtime/compss-engine.jar" >> ~/.bashrc
\end{lstlisting}

\colorComment{Attention: The \texttt{compss-engine.jar} is installed inside the COMPSs installation directory. If you have performed
a custom installation, the path of the package may be different.}


%
% SUBSECTION
%
\subsection{Jobs failed on method reflection}

When executing an application the main code gets stuck executing a task. Taking a look at the \texttt{runtime.log} users can check
that the job associated to the task has failed (and all its resubmissions too). Then, opening the \texttt{jobX\_NEW.out} or the
\texttt{jobX\_NEW.err} files users find the following error:

\begin{lstlisting}[language=bash]
[ERROR|es.bsc.compss.Worker|Executor] Can not get method by reflection
es.bsc.compss.nio.worker.executors.Executor$JobExecutionException: Can not get method by reflection
        at es.bsc.compss.nio.worker.executors.JavaExecutor.executeTask(JavaExecutor.java:142)
        at es.bsc.compss.nio.worker.executors.Executor.execute(Executor.java:42)
        at es.bsc.compss.nio.worker.JobLauncher.executeTask(JobLauncher.java:46)
        at es.bsc.compss.nio.worker.JobLauncher.processRequests(JobLauncher.java:34)
        at es.bsc.compss.util.RequestDispatcher.run(RequestDispatcher.java:46)
        at java.lang.Thread.run(Thread.java:745)
Caused by: java.lang.NoSuchMethodException: simple.Simple.increment(java.lang.String)
        at java.lang.Class.getMethod(Class.java:1678)
        at es.bsc.compss.nio.worker.executors.JavaExecutor.executeTask(JavaExecutor.java:140)
        ... 5 more
\end{lstlisting}

This error is due to the fact that COMPSs cannot find one of the tasks declared in the Java Interface. Commonly this is triggered by
one of the following errors:

\begin{itemize}
 \item The \textit{declaringClass} of the tasks in the Java Interface has not been correctly defined.
 \item The parameters of the tasks in the Java Interface do not match the task call.
 \item The tasks have not been defined as \textit{public}.
\end{itemize}


%
% SUBSECTION
%
\subsection{Jobs failed on reflect target invocation null pointer}

When executing an application the main code gets stuck executing a task. Taking a look at the \texttt{runtime.log} users can check
that the job associated to the task has failed (and all its resubmissions too). Then, opening the \texttt{jobX\_NEW.out} or the
\texttt{jobX\_NEW.err} files users find the following error:

\begin{lstlisting}[language=bash]
[ERROR|es.bsc.compss.Worker|Executor]
java.lang.reflect.InvocationTargetException
        at sun.reflect.NativeMethodAccessorImpl.invoke0(Native Method)
        at sun.reflect.NativeMethodAccessorImpl.invoke(NativeMethodAccessorImpl.java:57)
        at sun.reflect.DelegatingMethodAccessorImpl.invoke(DelegatingMethodAccessorImpl.java:43)
        at java.lang.reflect.Method.invoke(Method.java:606)
        at es.bsc.compss.nio.worker.executors.JavaExecutor.executeTask(JavaExecutor.java:154)
        at es.bsc.compss.nio.worker.executors.Executor.execute(Executor.java:42)
        at es.bsc.compss.nio.worker.JobLauncher.executeTask(JobLauncher.java:46)
        at es.bsc.compss.nio.worker.JobLauncher.processRequests(JobLauncher.java:34)
        at es.bsc.compss.util.RequestDispatcher.run(RequestDispatcher.java:46)
        at java.lang.Thread.run(Thread.java:745)
Caused by: java.lang.NullPointerException
        at simple.Ll.printY(Ll.java:25)
        at simple.Simple.task(Simple.java:72)
        ... 10 more
\end{lstlisting}

This cause of this error is that the Java object accessed by the task has not been correctly transferred and one or more of its fields
is null. The transfer failure is normally caused because the transferred object is not serializable. 

Users should check that all the object parameters in the task are either implementing the serializable interface or following 
the \textit{java beans} model (by implementing an empty constructor and getters and setters for each attribute).


%
% SUBSECTION
%
\subsection{Tracing merge failed: too many open files}

When too many nodes and threads are instrumented, the tracing merge can fail due to an OS limitation, namely: the maximum open files. This
problem usually happens when using advanced mode due to the larger number of threads instrumented. To overcome this issue users have two
choices. \textbf{First option}, use \textit{Extrae} parallel MPI merger. This merger is automatically used if COMPSs was installed with MPI
support. In Ubuntu you can install the following packets to get MPI support:

\begin{lstlisting}[language=bash]
sudo apt-get install libcr-dev mpich2 mpich2-doc
\end{lstlisting}

Please note that extrae is never compiled with MPI support when building it locally (with buildlocal command).

To check if COMPSs was deployed with MPI support, you can check the installation log and look for the following \textit{Extrae} configuration output:

\begin{lstlisting}[language=bash]
Package configuration for Extrae VERSION based on extrae/trunk rev. 3966:
-----------------------
Installation prefix: /gpfs/apps/MN3/COMPSs/Trunk/Dependencies/extrae
Cross compilation: no
CC: gcc
CXX: g++
Binary type: 64 bits

MPI instrumentation: yes
	MPI home: /apps/OPENMPI/1.8.1-mellanox
	MPI launcher: /apps/OPENMPI/1.8.1-mellanox/bin/mpirun
\end{lstlisting}

On the other hand, if you already installed COMPSs, you can check \textit{Extrae} configuration executing the script
\texttt{/opt/COMPSs/Dependencies/extrae/etc/configured.sh}. Users should check that flags \texttt{--with-mpi=/usr} and
\texttt{--enable-parallel-merge} are present and that MPI path is correct and exists. Sample output:

\begin{lstlisting}[language=bash]
EXTRAE_HOME is not set. Guessing from the script invoked that Extrae was installed in /opt/COMPSs/Dependencies/extrae
The directory exists .. OK
Loaded specs for Extrae from /opt/COMPSs/Dependencies/extrae/etc/extrae-vars.sh

Extrae SVN branch extrae/trunk at revision 3966

Extrae was configured with:
$ ./configure --enable-gettimeofday-clock --without-mpi --without-unwind --without-dyninst --without-binutils --with-mpi=/usr --enable-parallel-merge --with-papi=/usr --with-java-jdk=/usr/lib/jvm/java-7-openjdk-amd64/ --disable-openmp --disable-nanos --disable-smpss --prefix=/opt/COMPSs/Dependencies/extrae --with-mpi=/usr --enable-parallel-merge --libdir=/opt/COMPSs/Dependencies/extrae/lib

CC was gcc
CFLAGS was -g -O2 -fno-optimize-sibling-calls -Wall -W
CXX was g++
CXXFLAGS was -g -O2 -fno-optimize-sibling-calls -Wall -W

MPI_HOME points to /usr and the directory exists .. OK
LIBXML2_HOME points to /usr and the directory exists .. OK
PAPI_HOME points to /usr and the directory exists .. OK
DYNINST support seems to be disabled
UNWINDing support seems to be disabled (or not needed)
Translating addresses into source code references seems to be disabled (or not needed)

Please, report bugs to tools@bsc.es
\end{lstlisting}

\textbf{Disclaimer:} the parallel merge with MPI will not bypass the system's maximum number of open files, just distribute the
files among the resources. If all resources belong to the same machine, the merge will fail anyways.

\hfill

The \textbf{second option} is to increase the OS maximum number of open files. For instance, in Ubuntu add \texttt{ ulimit -n 40000 } just before
the start-stop-daemon line in the do\_start section.

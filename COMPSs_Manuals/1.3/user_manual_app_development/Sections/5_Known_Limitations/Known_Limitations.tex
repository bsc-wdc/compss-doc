\section{Known Limitations}
\label{sec:Known_Limitations}

Users must take into account that COMPSs is an on-going project. The \textit{Workflows and Distributed Computing} group 
at the \textit{BSC-CNS} is doing its best to maintain and enhance COMPSs; granting all its users a better support in
each new COMPSs release. 

Consequently, COMPSs is beeing continously developed and has several on-going projects. We are aware that the current COMPSs version
(1.3) has some limitations and we are working to solve them as soon as possible. Next, we provide a list of the major known
limitations: 
\begin{itemize}
 \item \textbf{Exceptions:} \newline The current COMPSs version is not able to catch exceptions raised from a task.
 
 \item \textbf{Services types:} \newline The current COMPSs version only supports SOAP based services that implement the WS interoperability
 standard. Supporting Big Services is a feature we are \textbf{not} planning to implment in the near future.
 
 \item \textbf{NIO workspaces:} \newline The persistent workers implementation (NIO) has a unique \textit{Working Directory} per worker. That
 means that tasks should not use hardcoded file names to avoid file colisions and tasks malfunction. We recommend to use files 
 declared as task parameters, to manually create a sandbox inside each task execution and/or to generate temporary random file names. 
\end{itemize}

If you are willing to know the COMPSs development state you can check out our webpage \url{http://compss.bsc.es} or 
contact us at \url{support-compss@bsc.es} .
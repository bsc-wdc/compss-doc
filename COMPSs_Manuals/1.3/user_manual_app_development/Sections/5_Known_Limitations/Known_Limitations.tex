\section{Known Limitations}
\label{sec:Known_Limitations}

The current COMPSs version (1.3) has the following limitations: 
\begin{itemize}
 \item \textbf{Exceptions:} \newline The current COMPSs version is not able to propagate exceptions raised from a task to the master. However, the runtime
 catches any exception and sets the task as failed.
 
 \item \textbf{Java tasks:} \newline Java tasks \textbf{must} be declared as \textbf{public}. Despite the fact that tasks can be
 defined in the main class or in other ones, we recommend to define the tasks in a separated class from the main method to force
 its public declaration.
 
 \item \textbf{Java objects:} \newline Objects used by tasks must follow the \textit{java beans} model (implementing an empty 
 constructor and getters and setters for each attribute) or implement the \textit{serializable} interface. This is due to the 
 fact that objects will be transferred to remote machines to execute the tasks.
 
 \item \textbf{Services types:} \newline The current COMPSs version only supports SOAP based services that implement the WS 
 interoperability standard. REST services are not supported.
 
 \item \textbf{Use of file paths:} \newline The persistent workers implementation has a unique \textit{Working Directory} per 
 worker. That means that tasks should not use hardcoded file names to avoid file collisions and tasks misbehaviours. We recommend to
 use files declared as task parameters, or to manually create a sandbox inside each task execution and/or to generate temporary 
 random file names. 
 
 \item \textbf{Tracing:} \newline The current version of the COMPSs tracing system slows down the application execution. Users
 running huge applications may experience a non-negligible overhead when using this feature. 
 
 \item \textbf{Intermediate files}: \newline Some applications may generate intermediate files that are only used among tasks
 and are never needed inside the master's code. However, COMPSs will transfer back these files to the master node at the end of the 
 execution. Currently, the only way to avoid transferring these intermediate files is to manually erase them at the end of the
 master's code. Users must take into account that this only applies for files declared as task parameters and \textbf{not} for files
 created and/or erased inside a task. 
 
 \item \textbf{Workers cache}: \newline Persistent workers maintain a cache to avoid transferring files and objects repeatedly. 
 However, this cache is \textbf{not} working for INOUT parameters (only works for IN and OUT parameters). For most applications, 
 if users are willing to exploit this cache, we recommend to convert INOUT parameters in \textbf{two} separated parameters: one 
 declared as IN parameter and the other declared as OUT parameter.
 
\end{itemize}


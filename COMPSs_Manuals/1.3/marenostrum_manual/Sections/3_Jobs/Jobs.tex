\section{COMPSs Jobs}
\label{sec:Jobs}

\subsection{Submiting COMPSs jobs}
COMPSs jobs can be easily submited by running the \textbf{enqueue\_compss} command. This command allows to configure any 
\textbf{runcompss} option and some particular queue options such as the queue system, the number of nodes, the wallclock time,
the master working directory, the workers working directory and number of tasks per node.

Next, we provide detailed information about the \textit{enqueue\_compss} command:
\begin{lstlisting}[language=bash]
$ enqueue_compss --help
Usage: /apps/COMPSs/1.3/Runtime/scripts/user/enqueue_compss 
         [queue_system_options] [COMPSs_options] 
         application_name application_arguments

* Options:
  General:
    --help, -h                              Print this help message
  
  Queue system configuration:
    - -exec_time=<minutes>                  Expected execution time of 
                                            the application (in minutes)
                                            Default: 10
                                            
    - -num_nodes=<int>                       Number of nodes to use
                                            Default: 2
                                            
    - -queue_system=<name>                  Queue system to use: 
                                            lsf | pbs | slurm
                                            Default: lsf
                                            
    - -tasks_per_node=<int>                 Maximum number of simultaneous
                                            tasks running on a node
                                            Default: 16
                                            
    - -master_working_dir=<path>            Working directory of the 
                                            application
                                            Default: .
                                            
    - -worker_working_dir=<name>            Worker directory. 
                                            Use: scratch | gpfs
                                            Default: scratch
                                            
    - -tasks_in_master=<int>                Maximum number of tasks that
                                            the master node can run as 
                                            worker. Cannot exceed 
                                            tasks_per_node.
                                            Default: 0
                                            
    - -network=<name>                       Communication network for 
                                            transfers:
                                            default | infiniband | data.
                                            Default: default
                                            
  Runcompss catched parameters:

    - -log_level=<level>, - -debug          Set the debug level: off | info 
                                            | debug
                                            Default: off
                                            
    - -tracing=<true|false>                 Enable tracing: true | false
                                            Default: false
                                            
    - -comm=<path>                          Class that implements the 
                                            adaptor for communications
                                            Default: integratedtoolkit.nio
                                            .master.NIOAdaptor
                                            
    - -library_path=<path>                  Non-standard directories to
                                            search for libraries (e.g. 
                                            Java JVM library, Python
                                            library, C binding library) 
                                            Default: .
                                            
    - -classpath=<path>                     Path for the application 
                                            classes / modules
                                            Default: .
  Runcompss delegated parameters:
                                            
  Runtime configuration options:
    - -project=<path>                       Path to the project XML file
                                            Default: /opt/COMPSs/Runtime/
                                            configuration/xml/projects
                                            /project.xml
                                            
    - -resources=<path>                     Path to the resources XML file
                                            Default: /opt/COMPSs/Runtime/
                                            configuration/xml/resources/
                                            resources.xml
                                            
    - -lang=<name>                          Language of the application
                                            (java/c/python)
                                            Default: java
                                            
    - -log_level=<level>, - -debug, -d      Set the debug level: off | 
                                            info | debug
                                            Default: off

  Tools enablers:
    - -graph=<bool>, - -graph, -g           Generation of the complete
                                            graph (true/false)
                                            When no value is provided it is 
                                            set to true
                                            Default: false
                                            
    - -tracing=<bool>, - -tracing, -t       Generation of traces 
                                            (true/false)
                                            When no value is provided it is 
                                            set to true
                                            Default: false
                                            
    - -monitoring=<int>, - -monitoring, -m  Period between monitoring 
                                            samples (milliseconds)
                                            When no value is provided it is
                                            set to 2000
                                            Default: 0

  Advanced options:
    - -comm=<path>                          Class that implements the 
                                            adaptor for communications
                                            Default: 
                                            integratedtoolkit.nio.master.
                                            NIOAdaptor
                                            
    - -library_path=<path>                  Non-standard directories to 
                                            search for libraries (e.g. 
                                            Java JVM library, Python 
                                            library, C binding library) 
                                            Default: .
                                            
    - -classpath=<path>                     Path for the application  
                                            classes / modules
                                            Default: .
                                            
    - -task_count=<int>                     Only for C/Python Bindings. 
                                            Maximum number of different 
                                            functions/methods invoked 
                                            from the application that 
                                            have been selected as tasks
                                            Default: 50
                                            
    - -uuid=<int>                           Preset an application UUID
                                            Default: Automatic random 
                                            generation

* Application name:
    For Java applications:   Fully qualified name of the application
    For C applications:      Path to the master binary
    For Python applications: Path to the .py file containing the main program
    
* Application arguments:
    Command line arguments to pass to the application. Can be empty. 
                                            
\end{lstlisting}

\subsection{Tracking COMPSs jobs}
When submitting a COMPSs job a temporal file will be created storing the job information. For example:
\begin{lstlisting}[language=bash]
$ enqueue_compss \
  --exec_time=15 \
  --num_nodes=3 \
  --queue_system=lsf \
  --tasks_per_node=16 \
  --master_working_dir=. \
  --worker_working_dir=gpfs \
  --lang=python \
  --log_level=debug \
  <APP> <APP_PARAMETERS>
  
Num Nodes:      3
Tasks per Node: 16
Tasks in Master:0
Master WD:      .
Worker WD:      gpfs
Exec-Time:      00:15
Network:        default
Library Path:   .
To COMPSs:      --lang=python --log_level=debug <APP> <APP_PARAMETERS>
Temp submit script is: /scratch/tmp/tmp.YPQKths559

$ cat /scratch/tmp/tmp.YPQKths559
#!/bin/bash
#
#BSUB -cwd . 
#BSUB -oo compss_3_%J.out
#BSUB -eo compss_3_%J.err
#BSUB -n 3
#BSUB -R"span[ptile=1]" 
#BSUB -J COMPSs
#BSUB -W 00:15 
...
\end{lstlisting}

In order to trac the jobs state users can run the following command:
\begin{lstlisting}[language=bash]
$ bjobs
JOBID  USER   STAT  QUEUE  FROM_HOST  EXEC_HOST  JOB_NAME  SUBMIT_TIME
XXXX   bscXX  PEND  XX     login1     XX         COMPSs    Month Day Hour
\end{lstlisting}

The specific COMPSs logs are stored under the \textit{~/.COMPSs/} folder; saved as a local \textit{runcompss} execution. For further 
details please check \textit{COMPSs User Manual: Application Execution} available at our webpage \url{http://compss.bsc.es} .

\subsection{Enabling COMPSs Monitor}
\subsubsection{Configuration}
As MareNostrum nodes are connection restricted, the better way to enable the \textit{COMPSs Monitor} is from the users local machine. 
To do so please install the following packages:
\begin{itemize}
 \item COMPSs Runtime
 \item COMPSs Monitor
 \item sshfs
\end{itemize}

For further details about the COMPSs packages installation and configuration please refer to the \textit{COMPSs Installation Manual} 
available at our webpage \url{http://compss.bsc.es} . If you are not willing to install COMPSs in your local machine please consider
to download our Virtual Machine available at our webpage. 
\newline

Once the packages have been installed and configured, users need to mount the sshfs directory as follows (\textit{MN\_USER} stands for 
your MareNostrum user and the \textit{TARGET\_LOCAL\_FOLDER} to the local folder where you wish to deploy the MareNostrum files):
\begin{lstlisting}[language=bash]
compss@bsc:~$ scp $HOME/.ssh/id_dsa.pub ${MN_USER}@mn1.bsc.es:~/id_dsa_local.pub
compss@bsc:~$ ssh MN_USER@mn1.bsc.es 
                  "cat ~/id_dsa_local.pub >> ~/.ssh/authorized_keys; 
                  rm ~/id_dsa_local.pub"
compss@bsc:~$ mkdir -p TARGET_LOCAL_FOLDER/.COMPSs
compss@bsc:~$ sshfs -o IdentityFile=$HOME/.ssh/id_dsa -o allow_other 
                   MN_USER@mn1.bsc.es:~/.COMPSs 
                   TARGET_LOCAL_FOLDER/.COMPSs
\end{lstlisting}

Whenever you wish to unmount the sshfs directory please run:
\begin{lstlisting}[language=bash]
compss@bsc:~$ sudo umount TARGET_LOCAL_FOLDER/.COMPSs
\end{lstlisting}

\subsubsection{Execution}
Access the COMPSs Monitor through its webpage (\url{http://localhost:8080/compss-monitor} by default) and login with the 
\textit{TARGET\_LOCAL\_FOLDER} to enable the COMPSs Monitor for MareNostrum. 

\section{Python Sample applications}
\label{sec:PythonSampleApps}

The first two examples in this section are simple applications developed in COMPSs to easily illustrate how to code,
compile and run COMPSs applications. These applications are executed locally and show different ways to take advantge
of all the COMPSs features. 

The rest of the examples are more elaborated and consider the execution in a cloud platform where the VMs mount a common 
storage on \textbf{/sharedDisk} directory. This is useful in the case of applications that require working 
with big files, allowing to transfer data only once, at the beginning of the execution, and to enable 
the application to access the data directly during the rest of the execution.

The Virtual Machine available at our webpage (\url{http://compss.bsc.es/}) provides a development environment with
all the applications listed in the following sections. The codes of all the applications can be found under the 
$/home/compss/workspace\_python/$ folder. 

%%%%%%%%%%%%%%%
%% SIMPLE
%%%%%%%%%%%%%%%
\subsection{Simple}
The Simple application is a Python application that increases a counter by means of a task. The counter is stored inside a file that 
is transfered to the worker when the task is executed. Next, we provide the main code and the task declaration:

\begin{lstlisting}[language=python]
def main_program():
    # Check and get parameters
    if len(sys.argv) != 5:
        usage()
        exit(-1)
    N = int(sys.argv[1])
    counter1 = int(sys.argv[2])
    counter2 = int(sys.argv[3])
    counter3 = int(sys.argv[4])

    # Initialize counter files
    initializeCounters(counter1, counter2, counter3)
    print "Initial counter values:"
    printCounterValues()

    # Execute increment
    for i in range(N):
        increment(FILENAME1)
        increment(FILENAME2)
        increment(FILENAME3)

    # Write final counters state (sync)
    print "Final counter values:"
    printCounterValues()
\end{lstlisting}

\begin{lstlisting}[language=python]
@task(filePath = FILE_INOUT)
def increment(filePath):
    # Read value
    fis = open(filePath, 'r')
    value = fis.read()
    fis.close()

    # Write value
    fos = open(filePath, 'w')
    fos.write(str(int(value) + 1))
    fos.close()
\end{lstlisting}

The simple application can be executed by invoking the runcompss command with the \textit{--lang=python} flag. The following lines provide
an example of its execution.

\begin{lstlisting}[language=bash]
compss@bsc:~$ cd ~/workspace_python/simple/
compss@bsc:~/workspace_python/simple$ runcompss --lang=python simple.py 1
\end{lstlisting}


\subsection{Increment}

\subsection{WordCount}

\subsection{Matmul}

\subsection{Neurons}

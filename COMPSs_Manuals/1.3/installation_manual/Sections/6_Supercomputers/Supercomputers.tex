\section{Supercomputers}
\label{sec:Supercomputers}

The COMPSs Framework can be installed in any SuperComputer by installing its packages as a normal distribution. The packages are
ready to be reallocated so the administrators can choose the right location for the COMPSs installation. \newline

However, if the administrators are not willing to install COMPSs through the packaging system, we also provide a \textbf{COMPSs zipped file} containing
a pre-build script to easily install COMPSs. For such a purpose please execute the following commands:
\begin{lstlisting}[language=bash]
 # Check out the last COMPSs release
 $ wget http://compss.bsc.es/repo/sc/stable/COMPSs_1.3.tar.gz

 # Unpackage COMPSs
 $ tar -xvzf COMPSs_1.3.tar.gz
 
 # Install COMPSs at your preferred target location
 $ cd COMPSs
 $ ./install <targetDir>
 
 # Clean downloaded files
 $ rm -r COMPSs
 $ rm COMPSs_1.3.tar.gz
\end{lstlisting}

The installation script will create a COMPSs folder inside the given $<targetDir>$, providing the final COMPSs installation 
under the $<targetDir>/COMPSs$ folder. Notice that, if such a folder already exists, it will be \textbf{automatically erased}.

~ \newline

To successfully run the installation script some dependencies must be installed in the target machine. Administrators must provide the correct
installation and environment of the following software:
\begin{itemize}
 \item Autotools
 \item BOOST
 \item Java 7 JRE
\end{itemize}

The following environment variables must be defined:
\begin{itemize}
 \item $JAVA\_HOME$
 \item $BOOST\_CPPFLAGS$
\end{itemize}

~ \newline

To allow users easily work with COMPSs this installation provides a \textit{compssenv} file that loads the COMPSs environment. Thus, after the 
installation process we recomend to source the $<targetDir>/COMPSs/compssenv$ into the users \textit{.bashrc}.


~ \newline
For further information about either the installation or the usage please check the \textit{README} file inside the package. 

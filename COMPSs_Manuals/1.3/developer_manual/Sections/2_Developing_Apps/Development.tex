\section{Developing COMPSs applications}
\label{sec:Developing}

In this section the steps to develop a COMPSs application will be illustrated; the sequential {\bf Simple application}
will be used to explain an application porting to COMPSs. The user is required to select a set of
methods, invoked in a sequential application, to be run as remote tasks on the available resources.

A COMPSs application is composed of three parts:
\begin{itemize}
 \item {\bf Main application code:} the code that is executed sequentially and contains the calls to the user-selected methods that will be executed on the Cloud.
 \item {\bf Remote methods code:} the implementation of the remote tasks.
 \item {\bf Java annotated interface:} It declares the selected methods to be run as remote tasks and metadata used to schedule the tasks.
\end{itemize}

The main application code (sequential) will have the name of the application, always starting with capital
letter, in this case will be {\bf Simple.java}. The Java annotated interface will be named as {\it application name+Itf.java} 
in this case will be {\bf SimpleItf.java}. And the code that implements the remote tasks will be called as
{\it application name + Impl.java}, in this case will be {\bf SimpleImpl.java}.

All code examples are in the {\bf /home/user/workspace/} folder of the development environment.

\subsection{Main application code}

In COMPSs the application is kept completely unchanged, i.e. no API calls need to be included in the main
application code in order to run the selected tasks on the nodes.

The COMPSs runtime is in charge of replacing the invocations to the user-selected methods with the
creation of remote tasks also taking care of the access to files from the main application code.

Let’s consider the Simple application example that takes an integer as input parameter and increases it by one unit.

The main application code of Simple app ({\bf Simple.java}) will be executed in a sequential way except the
{\bf increment()} method. COMPSs, as mentioned above, will replace at execution time the call to this method
generating a remote task on the remote node.

\begin{lstlisting}
package simple;

import java.io.FileInputStream;
import java.io.FileOutputStream;
import java.io.IOException;
import simple.SimpleImpl;

public class Simple {

  public static void main(String[] args) {
    String counterName = "counter";
    int initialValue = args[0];

//------------------------------------------------------------//
//Creation of the file which will contain the counter variable//
//------------------------------------------------------------//
    try {
       FileOutputStream fos = new FileOutputStream(counterName);
       fos.write(initialValue);
       System.out.println("Initial counter value is "
                           +initialValue);
       fos.close();
    }catch(IOException ioe) {
       ioe.printStackTrace();
    }
    
    //----------------------------------------------//
    //           Execution of the program           //
    //----------------------------------------------//
    
    %*{\bf SimpleImpl.increment(counterName);}*)
    
    //----------------------------------------------//
    //    Reading from an object stored in a File   //
    //----------------------------------------------//
    try {
       FileInputStream fis = new FileInputStream(counterName);
       System.out.println("Final counter value is "+fis.read());
       fis.close();
    }catch(IOException ioe) {
       ioe.printStackTrace();
    }
  }
}
\end{lstlisting}


\subsection{Remote methods code}

The following code is the implementation of the remote method of the {\it Simple} application ({\bf SimpleImpl.java})
that will be executed remotely by COMPSs.

\begin{lstlisting}
package simple;

import  java.io.FileInputStream;
import  java.io.FileOutputStream;
import  java.io.IOException;
import  java.io.FileNotFoundException;

public class SimpleImpl {
  public static void increment(String counterFile) {
    try{
      FileInputStream fis = new FileInputStream(counterFile);
      int count = fis.read();
      fis.close();
      
      FileOutputStream fos = new FileOutputStream(counterFile);
      fos.write(++count);
      fos.close();
    }catch(FileNotFoundException fnfe){
      fnfe.printStackTrace();
    }catch(IOException ioe){
      ioe.printStackTrace();
    }
  }
}
\end{lstlisting}


\subsection{Java annotated interface}

The Java interface is used to declare the methods to be executed remotely along with Java annotations that
specify the necessary metadata about the tasks. The metadata can be of three different types:

\begin{enumerate}
 \item For each parameter of a method, the data type (currently {\it File} type, primitive types and the {\it String} type are supported) and its directions (IN, OUT or INOUT).
 \item The Java class that contains the code of the method.
 \item The constraints that a given resource must fulfil to execute the method, such as the number of processors or main memory size.
\end{enumerate}

Here follows a complete and detailed explanation of the usage of the metadata:

\begin{itemize}
 \item {\bf Method-level Metadata:} for each selected method, the following metadata has to be defined:
       \begin{itemize}
         \item {\bf @Method:} Mandatory. It specifies the class that implements the method.
         \item {\bf @Constraints:} Mandatory. The user can specify the capabilities that a resource must have in order
               to run a method. The COMPSs runtime will create a VM (in a cloud environment), that fits the
               specified requirements in order to perform the execution.
               \begin{itemize}
                 \item Processor:
                       \begin{itemize}
                         \item {\bf processorCPUCount:} Number of required processors.
                       \end{itemize}
                \item Memory:
                       \begin{itemize}
                         \item {\bf memoryPhysicalSize:} Amount of GB of physical memory needed.
                       \end{itemize}
               \end{itemize}

       \end{itemize}

 \item {\bf Parameter-level Metadata (@Parameter):} for each parameter and method, the user must define:
       \begin{itemize}
        \item {\bf Direction:} {\it Direction.IN, Direction.INOUT or Direction.OUT}
        \item {\bf Type:} COMPSs supports the following types for task parameters:
              \begin{itemize}
               \item {\bf Basic types:} {\it Type.BOOLEAN, Type.CHAR, Type.BYTE, Type.SHORT, Type.INT, Type.LONG,
                     Type.FLOAT, Type. DOUBLE}. They can only have {\bf IN} direction, since primitive types in Java are
                     always passed by value.
               \item {\bf String:} {\it Type.STRING}. It can only have {\bf IN} direction, since Java Strings are immutable.
               \item {\bf File:} {\it Type.FILE}. It can have any direction (IN, OUT or INOUT). The real Java type associated
                     with a FILE parameter is a String that contains the path to the file. However, if the user specifies
                     a parameter as a FILE, COMPSs will treat it as such.
               \item {\bf Object:} {\it Type.Object}. It can have any direction (IN, OUT or INOUT).
              \end{itemize}
        \item {\bf Return type:} Any object, a basic type or a generic class object.
        \item {\bf Method modifiers:} the method has to be {\bf STATIC}.
       \end{itemize}
\end{itemize}

    
The Java annotated interface of the Simple app example (SimpleItf.java) declares the {\it Increment()} method
that will be executed remotely. The method implementation can be found in simple.SimpleImpl class and
needs a single input parameter, a string containing a path to the file counterFile. Besides, in this example
there are constraints on the minimum number of processors and minimum memory size needed to run the
method.

\begin{lstlisting}
package simple;

import  integratedtoolkit.types.annotations.Constraints;
import  integratedtoolkit.types.annotations.Method;
import  integratedtoolkit.types.annotations.Parameter;
import  integratedtoolkit.types.annotations.Parameter.Direction;
import  integratedtoolkit.types.annotations.Parameter.Type;

public interface SimpleItf {

  %*{\bf @Constraints(processorCPUCount = 1, memoryPhysicalSize = 0.3f) }*)
  %*{\bf @Method(declaringClass = "simple.SimpleImpl") }*)
  void increment(
      %*{\bf @Parameter(type = Type.FILE, direction = Direction.INOUT) }*)
      String file
  );

}
\end{lstlisting}


\subsection{Equivalent remote methods}
Since version 1.2, the COMPSs programming model allows developers to define sets of equivalent remote
methods. Thus, an invocation to any of the methods in the set might produce the execution of another
method of the set in the remote resource.

The coding of the application does not change, the remote methods are implemented as regular Java
methods and the main code of the application is a sequential code that contains calls to these methods. The
only component of the application that changes for defining a set of equivalent methods is the Java
annotated interface.

The programming model considers all the equivalent methods of a set as different implementations of the
same method. Therefore, the name and parameters of all the implementations must coincide; the only
difference is the class where the method is implemented. This is reflected in the attribute declaringClass of
the @Method annotation. Instead of stating that the method is implemented in a single class, the
programmer can define an array of declaring classes for the method.

The following code depicts an example where the developer sorts an integer array using two different
methods: merge sort and quick sort that are respectively hosted in the {\it packagepath.Mergesort} and
{\it packagepath.Quicksort} classes.

\begin{lstlisting}
%*{\bf @Method(declaringClass = \{ "packagepath.Mergesort", }*)
                         %*{\bf "packagepath.Quicksort" \}) }*)
                           
void sort(
    %*{\bf @Parameter(type = Type.OBJECT, direction = Direction.INOUT) }*)
    int[] array
);
\end{lstlisting}

As independent remote methods, the sets of equivalent methods might have common restrictions to be
fulfilled by the resource hosting the execution. Or even, each implementation might have specific constraints.
Through the @Constraints annotation, developers can specify the common constraints for a whole set of
methods. The following example states that for both sorting algorithms only one core is required to run the
method.

\begin{lstlisting}
%*{\bf @Constraints(processorCoreCount = 1) }*)
%*{\bf @Method(declaringClass = \{ "packagepath.Mergesort", }*) 
                         %*{\bf "packagepath.Quicksort"\}) }*)

void sort(
    %*{\bf @Parameter(type = Type.OBJECT, direction = Direction.INOUT) }*)
    int[] array
);
\end{lstlisting}

However, these sorting algorithms have different memory consumption, thus each algorithm might require a
specific amount of memory and that should be stated in the implementation constraints. For this purpose, the
developer can add a @Multiconstraints annotation containing the specific constraints for each
implementation. Since the Mergesort has a higher memory consumption than the quicksort, the following
example sets a requirement of 1 core and 2GB of memory for the mergesort implementation and 1 core and
500MB of memory for the quicksort.

\begin{lstlisting}
%*{\bf @Constraints(processorCoreCount = 1) }*)
%*{\bf @MultiConstraints(\{ }*)
        %*{\bf @Constraints(memoryPhysicalSize= (float)2.0), }*)
        %*{\bf @Constraints(memoryPhysicalSize= (float)0.5)\}) }*)
%*{\bf @Method(declaringClass = \{ "packagepath.Mergesort",  }*)
                         %*{\bf "packagepath.Quicksort"\}) }*)
void sort(
    %*{\bf @Parameter(type = Type.OBJECT, direction = Direction.INOUT) }*)
    int[] array
);
\end{lstlisting}
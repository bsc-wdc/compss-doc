\section{Building from sources}
\label{sec:Sources}
This section describes the steps to install COMPSs from the sources.

The first step is downloading the source code from the Git repository.
\begin{lstlisting}[language=bash]
$> git clone https://github.com/bsc-wdc/compss.git
$> cd framework
\end{lstlisting}

Then, you need to download the embedded dependencies from the git submodules.
\begin{lstlisting}[language=bash]
$ framework> ./submodules_get.sh
$ framework> ./submodules_patch.sh
\end{lstlisting}

Finally you just need to run the installation script. You have to options: 
For installing COMPSs for all the users run the following command. (root access is required)
\begin{lstlisting}[language=bash]
$ framework> cd builders/
$ builders> INSTALL_DIR=/opt/COMPSs/
$ builders> sudo -E ./buildlocal [options] ${INSTALL_DIR}
\end{lstlisting}
For installing COMPSs for the current user run the following command.
\begin{lstlisting}[language=bash]
$ framework> cd builders/
$ builders> INSTALL_DIR=$HOME/opt/COMPSs/
$ builders> ./buildlocal [options] ${INSTALL_DIR}
\end{lstlisting}

The different installation options can be found in the command help.
\begin{lstlisting}[language=bash]
$ framework> cd builders/
$ builders> ./buildlocal -h
\end{lstlisting}

\subsection{Post installation}
Once your COMPSs package has been installed remember to log out and back in again to end the installation process.

If you need to set up your machine for the first time please take a look at Section \ref{sec:Additional_Configuration} for a 
detailed description of the additional configuration. 
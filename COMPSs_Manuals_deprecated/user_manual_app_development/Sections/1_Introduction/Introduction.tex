\section{COMP Superscalar (COMPSs)}
\label{sec:Introduction}

COMP Superscalar (COMPSs) is a programming model which aims to ease the 
development of applications for distributed infrastructures, such as Clusters, 
Grids and Clouds. COMP Superscalar also features a runtime system that exploits 
the inherent parallelism of applications at execution time.

For the sake of programming productivity, the COMPSs model has four key 
characteristics:

\begin{itemize}
 
 \item  {\bf Sequential programming:} COMPSs programmers do not need to deal 
 with the typical duties of parallelization and distribution, such as thread 
 creation and synchronization, data distribution, messaging or fault tolerance. 
 thus eliminating most of the difficulties of concurrent/distributed programming.
 A task is a method or a service called from the application code that is intended to be spawned asynchronously and possibly run in parallel with other tasks on a set of resources, instead of locally and sequentially.
 
 \item  {\bf Infrastructure unaware:} COMPSs offers a model that abstracts the 
 application from the underlying infrastructure. Hence, COMPSs 
 programs do not include any detail that could tie them to a particular 
 platform, like deployment or resource management. This makes applications 
 portable between infrastructures with diverse characteristics.
 
 \item  {\bf Standard programming languages:} COMPSs natively supports Java applications, but also offers language bindings for Python and 
 C/C++ applications. 
 
 \item  {\bf No APIs:} In the case of COMPSs applications in Java, the model 
 does not require to use any special API call, pragma or construct in the 
 application; everything is standard Java syntax and libraries. As 
 regards the Python and C/C++ bindings, a small set of API calls should be used 
 on the COMPSs applications.

\end{itemize}

